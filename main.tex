\documentclass{article}

\usepackage[utf8]{inputenc}
%\usepackage[sort&compress]{natbib}
\usepackage{graphicx}
\graphicspath{ {images/} }
\usepackage[spanish]{babel}
\usepackage[acronym,shortcuts,nonumberlist,toc,section=section,numberedsection=false]{glossaries}
%\usepackage{cite} %para contraer las referencias
\usepackage{csquotes} % Por Biblatex http://tex.stackexchange.com/questions/229638/package-biblatex-warning-babel-polyglossia-detected-but-csquotes-missing
\usepackage{soul} % Para tachar texto
\usepackage{appendix}


% Paquetes para la bibliografía
\usepackage[
    style       =authoryear,
    sortcites   =true,
    bibencoding =ascii,
    sorting     =nyt,
    datelabel   =comp,
%    backref     =true, %para ver en el listado de la bibliografía donde la página donde se cita
    backend     =bibtex]
{biblatex}
%\DeclareLanguageMapping{spanish}{spanish-apa}
\renewcommand*{\nameyeardelim}{\addcomma\space} % Agrega una coma entre Apellido y año al citar.
\addbibresource{biblio.bib}
\addbibresource{normativa.bib}
\addbibresource{entrevistas.bib}
\addbibresource{sitiosweb.bib}

\makeglossaries


\input{acronimos}

\begin{document}

% Carátula
\setlength{\unitlength}{1 cm} %Especificar unidad de trabajo
\thispagestyle{empty}
\begin{picture}(0,0)
\put(0,0){\includegraphics[width=12cm,height=2.1cm]{banner_asap.png}}
%\put(0,0){\includegraphics[width=3cm,height=3cm]{asap.png}}
%\put(6,0.5){\includegraphics[width=4.75cm,height=1.5cm]{fce.png}}
\end{picture}
\\
\\
\begin{center}
\textbf{{\huge Universidad de Buenos Aires}\\[0.5cm]
{\large Facultad de Ciencias Económicas - Escuela de Estudios de Posgrado\\ Asociación Argentina de Presupuesto y Administración Financiera Pública}}\\[1.25cm]
{\Large Especialización en Gestión Pública por Resultados}\\[1cm]


\textbf{{\LARGE Los Sistemas de Información, Monitoreo y Evaluación y su vinculación con la Gestión por Resultados. El caso de la Administración Pública Nacional de Argentina}}\\[2.5cm]
{\Large por Matías Manuel Grandi}\\[0.2cm]
{\large Licenciado en Economía}\\[0.2cm]
{\large Cohorte 2012-2013}\\[1cm]


{\large Tutor de la Tesina}\\[0.2cm]
{\Large Norberto Carlos Perotti}\\[0.2cm]
{\large Licenciado en Economía}\\[2.2cm]
{\large Noviembre de 2015}
\end{center}
\newpage
\pagenumbering{Roman} % Arranca la numeración de páginas en números romanos
\setcounter{page}{1} % La numeración romana empieza con el número 1

\newpage
\section*{Evaluación}

\paragraph{I-  AUTOR DEL TRABAJO FINAL:} \mbox{}\\
Matías Manuel Grandi \\
Licenciado en Economía \\
2012-2013\\
Fecha de presentación: noviembre de 2015
\paragraph{II- TÍTULO DEL TRABAJO:} \mbox{}\\
Los Sistemas de Información, Monitoreo y Evaluación y su vinculación con la Gestión por Resultados. El caso de la Administración Pública Nacional de Argentina
\paragraph{III- COMENTARIOS:} \mbox{}\\
\paragraph{A. Contenido abreviado} \mbox{} \\
Este escrito es un documento cuyo contenido realiza una descripción de los sistemas de monitoreo y evaluación vigentes en la Administración Pública Nacional analizando la potencialidad de la integración de las tareas de los diferentes sistemas para lograr una más efectiva utilización de las herramientas de la evaluación. Ello fundamentado en la importancia de esa tarea para poder obtener mejoras concretas en las acciones de planificación en las Jurisdicciones y Entidades del Gobierno y en la ejecución de las políticas por parte de esos organismos.\\
\paragraph{B. Comentarios específicos:} \mbox{} \\
    \begin{itemize}
        \item Sobre su pertinencia y alcance estimo que el trabajo presenta una síntesis de los principales sistemas de monitoreo y evaluación que abarcan un amplio espectro de esa actividad en la APN.
        \item El análisis es el apropiado a los efectos de sustentar las conclusiones a las que arriba el autor.
        \item El desarrollo es fluido y no hay información redundante y repetida que obstaculice la lectura y la comprensión de los objetivos que se plantea el autor.
        \item Sobre la bibliografía utilizada: Presenta una extensa bibliografía referida a las políticas públicas, la gestión por resultados, y las implementación de los sistemas de monitoreo y evaluación, con un detallado análisis normativo, y entrevistas personales a los encargados de los organismos de mayor trascendencia en la materia.
        \item La organización y escritura del trabajo son adecuadas y concluyen en el buen resultado de un texto que resulta interesante en su desarrollo y sostiene adecuadamente las conclusiones a la que arriba.
        \item El trabajo es muy bueno en la fundamentación del porqué de la necesidad de una integración de las áreas de evaluación de la APN a los efectos de la mejora de la gestión de las políticas públicas por parte de los que planifican y ejecutan las mismas.
    \end{itemize}
\paragraph{IV- NOTA PROPUESTA} \mbox{} \\
Se propone la nota de 8 (ocho) puntos.
 
 
\vspace{3cm}
Lic. Norberto Carlos Perotti

\vspace{0.1cm}
Buenos Aires, 12 de noviembre de 2015

%\maketitle
\renewcommand{\acronymname}{Índice de Acrónimos}

% Tabla de Contenidos
\newpage
\tableofcontents

% Indice de acrónimos
\newpage
\printglossary[type=\acronymtype]
\printglossary
\newpage

% Cuerpo principal

\newpage

\section*{Prólogo}
\addcontentsline{toc}{section}{Prólogo}
La presente tesina se realizó en el marco de la Especialización en Gestión Pública por Resultados dictada por la Facultad de Ciencias Económicas de la Universidad de Buenos Aires y la Asociación de Presupuesto y Administración Financiera Pública.

Tanto la \ac{gpr} como los sistemas de información, monitoreo y evaluación han tomado gran importancia en la \ac{apn}, que busca modernizar sus sistemas de gestión. Por tal motivo, el tema elegido cobra una gran importancia a la hora de analizar las herramientas utilizadas por el Sector Público.

A efectos de lograr un acabado análisis de las herramientas generadas a partir de la impronta de la \ac{gpr}, se toma como punto de partida aquellos organismos, programas o proyectos creados a partir de la denominada Reforma del Estado, en la década del '90, y profundizada en los últimos 10 años.

Debido a mi trayectoria profesional en varios de los organismos que impulsan estás nuevas actividades, la tesina abunda en información obtenida en entrevistas realizadas de manera personal, un detallado análisis normativo y en una seleccionada bibliografía teórica.

Por último, un especial agradecimiento al tutor de esta tesina, el Lic. Norberto Perotti, quien apoyó en todo momento y firmemente la elaboración del presente documento.
\pagenumbering{arabic} % Arranca la numeración de páginas en números arabicos
\newpage
\section{Introducción}\label{introduccion}

Los sistemas de información, monitoreo y evaluación del sector público han tomado gran relevancia como herramientas para la toma de decisiones en las ultimas dos décadas. La necesidad de contar con información sobre el rumbo de la gestión y el desempeño de las políticas públicas volvió a centrar la atención de los decisores en estas actividades, lo que provocó el desarrollo de nuevas áreas, programas y organismos para cumplir con esas funciones.

Por otro lado, estos sistemas forman parte de los pilares fundamentales de la \ac{gpr}, que aparecen generalmente citados en la bibliografía sobre el tema. La \ac{gpr} es una herramienta de gestión que se instaló en las agendas de los gobiernos desde mediados de la década de los '80, con el surgimiento de la \ac{ngp}, principalmente en Reino Unido, Australia y Nueva Zelanda. En los países de América Latina, con los procesos de reforma de la década de los '90, se comenzaron a desarrollar instrumentos para mejorar la gestión del Estado ``hacia adentro''. La \ac{gpr} no tiene un modelo unívoco de implementación. Las características propias de la normativa, la cultura organizacional y las formas de gobierno transforman cada burocracia en un caso único.

En Argentina, en la \ac{apn}, han existido diferentes intentos para dotar a la gestión de información de calidad, como para implantar sistemas de evaluación dentro del ciclo de gestión de las políticas públicas. El éxito de cada una de estas instancias ha dependido, en gran parte, del impulso político, la capacidad técnica, y la fortaleza institucional. En el Poder Ejecutivo nacional de Argentina, las capacidades institucionales para diseñar, planificar, implementar, monitorear y evaluar políticas difieren entre ministerios y agencias de gobierno, provocando niveles heterogéneos en la calidad de las intervenciones. Esto no implica que no exista un sistema de \ac{gpr}, pero si que las diferentes instancias de la misma estén con diversos niveles de implementación y funcionamiento.

El presente documento busca indagar sobre el funcionamiento de los sistemas, programas u organismos creados a la luz del procesos iniciado en los '90 con la denominada Reforma del Estado, y su vinculación con la \ac{gpr}. Dentro los nuevos instrumentos generados por dicha Reforma, se destacan los sistemas de información, monitoreo y evaluación de programas y políticas. Por este motivo, no se hará mención de otras importantes innovaciones como las referidas a la técnica presupuestaria, de importante avance en los últimos 20 años.

En la Sección \ref{gpr}, se presenta un definición conceptual de la \ac{gpr},  la cronología de su desarrollo en en la \ac{apn}, con un detallado análisis de la principal normativa vinculada al proceso y la descripción del \ac{prodev}, el programa que buscar apoyar la implementación de la \ac{gpr} en Argentina. En la Sección \ref{sistemas} se describen los principales sistemas de información, monitoreo y evaluación generados en la \ac{apn}, y en la Sección \ref{articulacion} se caracteriza el funcionamiento de los organismos descritos en el punto anterior, focalizándose en las necesidades que la herramienta de la \ac{gpr} demandan de los mismos. Por último, se elabora una conclusión en función del análisis realizado.
\newpage
\section{Sobre la Gestión por Resultados} \label{gpr}
\subsection{Marco conceptual}

La \ac{gpr} forma parte de la agenda pública desde que se produjeron importantes transformaciones en la relación entre el Estado y la sociedad en las últimas décadas. Sin embargo, no existe ni un modelo ni un modo de implementación unívoco de los distintos instrumentos y es, precisamente, en estas diferencias donde se expresan el tipo de Estado que se busca y el compromiso que el Estado asume con los ciudadanos.

El inicio de este proceso debe rastrearse en la década de 1980, cuando se pusieron en marcha reformas administrativas-gerencialistas en los sectores públicos de países como Nueva Zelanda, Reino Unido y Australia y cuyo eje fue la modificación de la principal herramienta que tiene el Estado para asignar recursos, el sistema de presupuesto público, con el objeto de pasar de un sistema orientado a los gastos a uno orientado a la búsqueda de resultados \parencite{arellano1999}.

El paradigma de la \ac{ngp} promovió la modernización de las administraciones públicas. En América Latina, a posteriori de las reformas orientadas a reducir el tamaño del Estado y sus funciones, comenzó la preocupación por reformar el Estado ``hacia adentro'' \parencite{oszlako1999}. Estos procesos se sustentaron también en el diagnóstico realizado por los organismos multilaterales de crédito que a partir del año 1997 comenzaron a promover una serie de medidas destinadas a mejorar la administración pública, fundándose en la necesidad de incrementar las capacidades institucionales y de gestión estatal como premisa para lograr un buen funcionamiento de los mercados.

Como todo paradigma, la \ac{ngp} presenta matices en su interior. En su versión más ortodoxa plantea transpolar técnicas de management o gerenciamiento, del sector privado al sector público, presentadas como neutrales políticamente y aplicables a cualquier contexto y organización, con la finalidad de reorientar el servicio público hacia la demanda, bajo criterios de eficiencia, eficacia y economía.

Con todo, la necesidad de planificar, obtener resultados y sustentar el mecanismo de toma de decisiones en un diagnóstico científicamente fundado de la situación existente, no es ajena a la tradición de la estatalidad.
Según \citeauthor{garcialopez2010} (\citeyear{garcialopez2010}), la \ac{gpr} es una estrategia de gestión pública que conlleva tomar decisiones sobre la base de información confiable acerca de los efectos que la acción gubernamental tiene en la sociedad. Varios países desarrollados la han adoptado para mejorar la eficiencia y la eficacia de las políticas públicas. En América Latina y el Caribe, señalan estos autores, los gobernantes y administradores públicos muestran un interés creciente en esta herramienta de gestión. En tal sentido, la \ac{gpr} es una estrategia integral que toma en cuenta los distintos elementos del ciclo de gestión (planificación, presupuesto, gestión financiera, gestión de proyectos, monitoreo y evaluación), subrayándose el papel que desempeñan estos elementos en la creación de valor público.

La consolidación de este nuevo modelo de gestión pública, requiere una modificación de la cultura administrativa que admita una definición clara de los objetivos de la organización, focalizados en sus aspectos sustanciales y no como procesos administrativos formales, de manera que la evaluación de la gestión pública se realice a través del cumplimiento de metas más que a partir, solamente, del respeto a reglas que, en muchas ocasiones, son auto-referidas por la burocracia. Se espera que de esa manera se fortalezca la capacidad de la administración pública para aprender de su desempeño y mejorar continuamente la prestación de servicios públicos. 

Siguiendo a la clasificación elaborada por el \ac{sep} \parencite{sepprodev}, la \ac{gpr} divide al ciclo de gestión en cinco pilares indispensables para que el proceso de creación de valor público esté orientado a lograr resultados:

    \begin{enumerate}
        \item planificación para resultados, 
        \item \ac{ppr}, 
        \item gestión financiera, auditoría y adquisiciones, 
        \item gestión de programas y proyectos y 
        \item monitoreo y evaluación. 
    \end{enumerate}

\subsection{La Gestión por Resultados en Argentina}

La implementación de la \ac{gpr} en Argentina se remonta a fines de la década del '80 y principios de la década del '90. En esa época tuvieron lugar algunas experiencias en torno al planeamiento estratégico y a la reorganización de estructuras de organismos públicos, que en general no fueron formuladas en forma coordinada, generalizada ni sistemática. Principalmente estuvieron marcadas por esfuerzos atomizados de algunos organismos, que no fueron acompañados por procesos impulsados por la gestión política del país.

La formalización de todos estos procesos vio la luz recién en el año 2001, cuando por medio del \citeauthor{decreto103} (\citeyear{decreto103}) se creó el Plan para la Modernización del Estado, que impulsó de manera formal la \ac{gpr} en la \ac{apn}.
Como se describe en la Sección \ref{marco-normativo}, este decreto establecía la elaboración de indicadores de gestión, Acuerdos Programa y sistemas de incentivos por desempeño institucional.

Este proceso, iniciado al menos formalmente, se vio interrumpido por la grave crisis que atravesó el país durante ese mismo año y los avances en la materia fueron escasos, a pesar que la legislación continúe vigente a la fecha.

Mas allá de algunas acciones aisladas, la \ac{gpr} como sistema recién volvió a tomar relevancia en el año 2009,  con la \textcite{cartaprodev} firmada entre el \ac{bid} y el \ac{mecon} (ratificada por \textcite{decreto1673}), que formalizaba la cooperación técnica no reembolsable a para la Cuenta A del \ac{prodev}.



    \subsubsection{Marco Normativo}\label{marco-normativo}

El marco normativo necesario para la aplicación de la \ac{gpr} está conformado, por un lado, por las normas referidas a la Reforma del Estado, y por el otro, por las vinculadas al sistema presupuestario, financiero y de control. En esta sección se enumeran y describen brevemente los decretos y normativas de menor rango que formulan el Plan para la Modernización del Estado, estableciendo explícitamente como uno de sus ejes principales a la \ac{gpr}. Posteriormente se presentan las normas referidas a las estructuras y procesos necesarios para su implementación.

\paragraph{Normativa vinculada a la Reforma del Estado:} Dada la naturaleza de la \ac{gpr}, los procesos que abarca, y la articulación jurisdiccional que plantea, la sanción de una ley por sí misma no determina su exitosa implementación en la \ac{apn}. Por consiguiente, la normativa citada a continuación es un compendio que rescata la legislación que busca modernizar y articular los pilares que forman parte del sistema a través de la incorporación de nuevas metodologías y técnicas, para generar un todo coordinado donde el sistema en general tienda a transformar la gestión de la \ac{apn} en una orientada en resultados.

    \begin{enumerate}
        \item \citeauthor{ley24156} (\citeyear{ley24156}): La \citetitle{ley24156} fue uno de los principales instrumentos normativos a partir del cual se inició el profundo proceso de reforma estatal.
        \item \textcite{ley24629}: dio comienzo a la Segunda Reforma del Estado, estableciendo algunas normas para la ejecución del presupuesto de la \ac{apn} -que complementaban lo previsto en la \citeauthor{ley24156} -como así también nuevos lineamientos con el objetivo de mejorar el funcionamiento y la calidad de los servicios prestados por el Estado. En esta normativa se especificaban también los lineamientos básicos a considerar para la elaboración de los planes estratégico.
        \item \textcite{ley25512}: denominada Ley de solvencia fiscal y Calidad del Gasto Público, establecía las medidas a partir de las cuales se deberían ajustar los poderes del estado nacional para administrar los recursos públicos. Estas medidas se centraban en:
            \begin{enumerate}
                \item Formulación del presupuesto general de la \ac{apn},
                \item Eficiencia y calidad de la gestión pública,
                \item Programa de evaluación de calidad del gasto,
                \item Presupuesto plurianual,
                \item Información pública y de libre acceso,
                \item Creación del fondo anticíclico fiscal.
            \end{enumerate}
        Esta ley autoriza al Jefe de Gabinete de Ministros a partir del año 2000, a realizar Acuerdos Programas con las Unidades Ejecutoras de programas presupuestarios, para avanzar en el proceso de reforma del Estado, aumentar la eficiencia y lograr mejorar la calidad de gestión. Así es como aparece por primera vez en la legislación la figura de Acuerdo Programa.
        \item \textcite{decreto103}: Aprueba el Plan Nacional de Modernización del Estado. Esta normativa proponía una transformación centrada en la elaboración de indicadores de gestión, Acuerdos Programa y sistemas de incentivos por desempeño institucional. Este Decreto puede considerarse como el antecedente fundacional en cuanto a la introducción de la \ac{gpr} en el ámbito de la \ac{apn}. Establece un sistema de incentivos otorgado en función del cumplimiento de objetivos prefijados. Se otorga facultades para que los organismos realicen modificaciones en sus estructuras organizativas.
    \end{enumerate}	

\paragraph{Normativa vinculada a Estructuras y Procesos:}
Además de las leyes generales que dan el marco para la reforma del Estado en la Argentina y que proporcionan el encuadre y direccionamiento de las políticas que se pueden vincular a la \ac{gpr}, existen otra normas que dan lugar a la formación de estructuras y el desarrollo de procesos en dicha materia.

    \begin{enumerate}
        \item \textcite{decreto558}: Crea la Unidad de Reforma y Modernización del Estado, que constituyó uno de los primeros antecedentes institucionales relevantes en cuanto a la introducción del planeamiento estratégico como metodología para la modernización del aparato público. Se conformó en el ámbito de la \ac{jgm}. Su función esencial consistía en concluir el proceso de reforma del Estado y elaborar un programa de modernización para el mismo.
        \item \textcite{decreto928}: Establece iniciar una transformación profunda e incorporar el planeamiento estratégico como método en cinco organismos descentralizados considerados clave: \acs{dgi}, \acs{anssal}, \acs{anses}, \acs{inssjp} y \acs{ana}, lo cual implicaba el diseño de un plan estratégico para cada uno de ellos. Este decreto instituye la obligatoriedad del diseño de dichos planes para los organismos descentralizados y postula un nuevo rol del estado, teniendo como eje principal la orientación al ciudadano, la medición de resultados y la jerarquización y participación de los recursos humanos.
        \item \textcite{decreto229}. Crea del Programa Carta Compromiso con el Ciudadano. El Programa tiene como principal finalidad mejorar la relación de la \ac{apn} con los ciudadanos, especialmente a través de la calidad de los servicios que ella brinda. El punto de partida para su implementación lo constituye la decisión de los organismos de comenzar a concebir y desarrollar los servicios públicos, con la mirada de quienes los utilizan o reciben. La Carta Compromiso con el Ciudadano intenta enmarcar la relación entre los ciudadanos y los organismos públicos, por lo que podría considerarse uno de los elementos materiales esenciales de la Reforma del Estado.
        \item \textcite{decreto673}: Crea la Secretaría de Modernización del Estado, organismo responsable de articular e implementar las medidas establecidas a partir de la llamada Segunda Reforma del Estado.
        \item Decreto 992/01: establece la creación de las Unidades Ejecutoras de Programas, las que tendrán a su cargo –bajo la responsabilidad de los Gerentes de Programa- la ejecución de Programas específicos en relación directa con actividades vinculadas a la obtención de resultados hacia la población objetivo. El mismo específica en su artículo 6º ``El Compromiso de Resultados de Gestión deberá especificar los resultados a alcanzar por el responsable de la Unidad Ejecutora de Programa, los recursos que se pondrán a disposición y los indicadores objetivos de desempeño, de manera convergente con las metas físicas establecidas para el programa en el Presupuesto Nacional, en el caso que las hubiera. Dichos indicadores integrarán el Sistema de Evaluación de Resultados, en el que también se especificará la periodicidad de las evaluaciones parciales de la labor de los Gerentes de Programa. Cuando la naturaleza de la oferta pública lo permita, los resultados a alcanzar podrán ser acordados en diferentes escenarios o niveles de esfuerzo en función de la dotación de recursos reales y financieros que se disponga.''
        \item \textcite{decreto21}: Con el propósito de consolidar las instituciones, fortaleciendo a la \ac{apn} y al Estado, el Gobierno decidió a fines de 2007 elevar la hasta entonces Subsecretaría de la \ac{jgm}, denominándola Secretaría de la Gestión Pública. El organismo que se torna central en cuanto a la promoción de la \ac{gpr} en esta etapa es la \ac{onig}
        \item \textcite{decreto22}: Modifica la estructura orgánica de la \ac{jgm} y faculta a la Subsecretaría de Gabinete y Coordinación Administrativa para entender en el proceso de monitoreo y evaluación de la ejecución de las políticas públicas; coordinar con los distintos organismos de la Administración Pública Nacional la articulación de los sistemas de evaluación sectoriales; desarrollar un sistema de seguimiento de los programas de gobierno, estableciendo indicadores claves de las políticas prioritarias, para la toma de decisiones; establecer y coordinar un canal permanente de intercambio con los sistemas de información y monitoreo de planes para posibilitar la correcta evaluación del impacto de la implementación de las políticas de la Jurisdicción.
        \item Pero tal vez la medida más precisa, y que  resume la voluntad de articulación de los diversos actores de la \ac{apn} con injerencia en los ejes fundamentales de la \ac{gpr} es la \textcite{cartaprodev} firmada entre el \ac{bid} y el \ac{mecon} en diciembre de 2009. ``A efectos de facilitar la ejecución, coordinación y sostenibilidad del Programa, se crea formalmente un Comité Asesor integrado por el Subsecretario de Evaluación Prespuestaria y el Subsecretario de Gestión Pública de la \ac{jgm} y el Subsecretario de Presupuesto del \ac{mecon}. Dicho Comité tendrá, entre otras, las siguientes funciones: 
        \begin{itemize}
            \item definir los grandes lineamientos, conjuntamente con el Coordinador Técnico del Programa; 
            \item coordinar la estrategia general del Programa; 
            \item adquirir conocimiento del contenido y alcance de los \acrshort{poa}, hacer seguimiento al estado de avance de la operación y recomendar ajustes; 
            \item comunicarse con el nivel político estratégico en cada Ministerio para asegurar el apoyo necesario para desarrollar e implementar las acciones vinculadas a la gestión por resultados; y
            \item velar para que el Programa cuente con los recursos de contrapartida suficientes para su oportuna ejecución y para que cumpla con sus objetivos.'' 
        \end{itemize}
        Este comité se encuentra conformado por los tres pilares fundamentales de la \ac{gpr}: Planificación (Subsecretaría de Gestión y Empleo Público), Presupuesto (Subsecretaría de Presupuesto) y Coordinación y Evaluación (Subsecretaría de Evaluación del Presupuesto Nacional).
        \item \textcite{resolucion416} de \ac{jgm}: Crea en la órbita de la Secretaría de Gabinete y Coordinación Administrativa de la \ac{jgm}, el ``Programa de Evaluación de Políticas Públicas'', destinado a contribuir al proceso de institucionalización de la evaluación de políticas públicas en la Administración Pública y potenciar las capacidades para su desarrollo con miras a mejorar la gobernabilidad, la calidad de las políticas y los resultados en la gestión de los asuntos públicos.
    \end{enumerate}	


    \newpage
\subsection{¿Qué es el PRODEV?} \label{prodev}

El \ac{prodev} es una iniciativa del \ac{bid}, que  abarca una serie de acciones específicas que procuran fortalecer la efectividad de los gobiernos de la región, a fin de que puedan alcanzar mejores resultados en sus intervenciones de desarrollo.
Según el sitio web del \ac{bid} \footnote{\citeurl{bid}}:

\begin{quote}
\small ¿Cuál es el objetivo del \ac{prodev}? El objetivo principal del \ac{prodev} es apoyar a los países miembros prestatarios, interesados en mejorar la gestión del sector público (incluido el diseño, la ejecución, el seguimiento y la evaluación de políticas, estrategias, programas y proyectos), de una forma coherente con la asignación y el uso eficientes de los recursos de los ministerios y departamentos centrales (finanzas, planificación y presupuesto), ministerios sectoriales (salud, infraestructura y educación) y gobiernos subnacionales (estatales, provinciales, municipales y locales).  
\end{quote}

A efectos de institucionalizar el programa, el gobierno argentino crea en 2009 una cooperación técnica no reembolsable y una \ac{ueprodev} en la entonces Secretaría de Evaluación Presupuestaria de la \ac{jgm}, por la cual el \ac{bid} financia la implementación del \ac{prodev} en Argentina. Sus objetivos específicos eran:  
    \begin{itemize}
        \item analizar la situación de las distintas áreas clave para una gestión por resultados y, con base en la información relevada, preparar un diagnóstico y un plan de acción para el país; y 
        \item sensibilizar y capacitar a directivos y técnicos del sector público con el fin de generar el cambio cultural e institucional necesario para la implantación de la gestión pública por resultados. 
    \end{itemize}

Cumpliendo con su segundo objetivo específico, la \ac{ueprodev} se contacta con diversos organismos de la \ac{apn} para plantear una serie de acciones en conjunto con el fin de apoyar la implementación de la \ac{gpr} en los mismos, buscando una mayor eficacia y calidad de las Políticas Públicas.

La \ac{jgm} ejecuta el Programa a través de una unidad ejecutora creada al efecto en la Subsecretaría de Evaluación del Presupuesto Nacional, siendo la Secretaría de Evaluación Presupuestaria de \ac{jgm} la coordinadora del Programa.

Con el fin de facilitar la ejecución, coordinación y sostenibilidad del Programa, se creó formalmente un Comité Asesor integrado por el Subsecretario de Evaluación del Presupuesto Nacional y el Subsecretario de Gestión y Empleo Público de la \ac{jgm} y el Subsecretario de Presupuesto del \ac{mecon}.

A su vez, coordina la acción conjunta con otros actores estratégicos de la \ac{apn}, de acuerdo a cada uno de los los pilares fundamentales que impulsa y promueve la \ac{ueprodev}:

    \begin{enumerate}
        \item En materia de Planificación Estratégica, la \ac{onig} de la Subsecretaría de Gestión y Empleo Público de la Jefatura de Gabinete de Ministros.
        \item en relación al Presupuesto, la oficina \ac{onp} de la Subsecretaría de  Presupuesto del \ac{mecon}
        \item respecto a la evaluación, el Programa de Evaluación de Políticas Públicas, ejecutado conjuntamente entre la Secretaría Evaluación Presupuestaria y la Secretaría de Gabinete y Gestión Administrativa de la \ac{jgm}. 
    \end{enumerate}

La implementación del \ac{prodev} en Argentina fue diseñado para que su ejecución se cumpla en dos etapas. La primera, denominada Cuenta A, se comenzó a ejecutar durante el año 2011, finalizando su ejecución en diciembre 2012. Posteriormente, la Cuenta B fue aprobada en diciembre 2013 (\cite{cartaprodevb}), y actualmente se encuentra en proceso de ejecución. A continuación se detallan, algunas de las principales actividades realizadas por el Programa en pos de apoyar la implementación de la \ac{gpr} según se describen en \citetitle{informegestionprodev} (\citeyear{informegestionprodev}).

\subsubsection{PRODEV Cuenta A: Actividades}

La cuenta A tenía como foco la sensibilización, capacitación y  elaboración de un diagnóstico sobre la \ac{gpr} en el país. Las actividades desarrolladas a tal fin fueron:
    \begin{itemize}
        \item Jornadas ASAP de Capacitación sobre Planeamiento Estratégico (Diciembre 2012)
%        \item Taller de intercambio de experiencias para los funcionarios que participan en la articulación de las acciones de las áreas que intervengan en la asistencia técnica en la implementación de \ac{gpr}. 25 participantes.
        \item Taller de sensibilización en los organismos y jurisdicciones.
        \item Diagnóstico Estado de Situación \ac{gpr} en Argentina. 
        \item Coordinación Misión \ac{sep}: La unidad ejecutora \ac{prodev} coordinó las entrevistas y apoyó el relevamiento del Sistema de Evaluación Prodev 2012 del \ac{bid}
        \item Metodología para una Estrategia de Intervención: Documento que desarrolla una metodología de intervención organizacional coordinada por parte de los pilares fundamentales de la GpRD, articulando la planificación, el presupuesto y la evaluación.
        \item Articulación Planificación Estratégica-Operativa-SISEG: A partir del trabajo entre los directivos y técnicos de las áreas de planificación, monitoreo y la  Unidad Ejecutora del Prodev se consensuaron un conjunto de acuerdos de articulación para el trabajo conjunto a replicarse en organismos de la APN.
        \item \acrshort{incucai}: Se completó exitosamente la formulación del plan estratégico y operativo, y se articuló con el \ac{mecon} para la formulación de indicadores de resultados y revisión de la estructura presupuestario.   
        \item Articulación con MTEySS: dado que el organismo viene desarrollando hace varios años importantes avances en planificación estratégica y operativa, gestionó la cesión del nuevo tablero de control (\ac{ssg}) para el seguimiento de sus indicadores.
        \item Capacitaciones CEPAL y AECID: Técnicos \ac{prodev} realizaron capacitaciones en \ac{cepal} (Chile) sobre “Matriz de Marco Lógico” y en la \ac{aecid} (Santa Cruz de la Sierra) sobre “Evaluación de Políticas Públicas”.
        \item Conferencia Internacional: Participación del Subsecretario en la “Third International Conference on National Evaluation Capacities” (San Pablo, 2013) organizada por \acrshort{pnud} y el Gobierno del Brasil para presentar los avances del \ac{prodev} en Argentina y documentos teóricos de articulación plan-presupuesto-evaluación.
        \item Programa de Evaluación de Políticas Públicas: La Unidad Ejecutora del \ac{prodev} formó parte de la elaboración y la gestión del Programa de Evaluación de Políticas Públicas, junto con la Subsecretaría de Gestión y Empleo Público y la Subsecretaría de Evaluación de Proyectos con Financiamiento Externo. El programa tiene como objetivo principal institucionalizar los procesos de evaluación de políticas públicas en la Administración Pública Nacional y potenciar las capacidades para su desarrollo.
    \end{itemize}

\subsubsection{PRODEV Cuenta B: Actividades}
La cuenta B del \ac{prodev} se encuentra actualmente comenzando su ejecución y tiene planteadas las siguientes actividades, según se explicita en el \citetitle{poamedianoplazo}:
\paragraph{Componente I. Sistema Nacional de Planificación} \mbox{}\\

Tiene como objetivo dotar a la planificación en la \ac{apn} de mayor capacidad estratégica y operativa
Resultados esperados: dotar a la \ac{apn} con una propuesta de marco conceptual, institucional, normativo y metodológico para la elaboración de planes estratégicos y operativos, vinculados con la programación presupuestal; e implantar instrumentos y sistemas de planificación en un conjunto de ministerios de la APN

Los productos esperados son: 
    \begin{itemize}
        \item Diseño conceptual, institucional y normativo del Sistema Nacional de Planificación, recogiendo las experiencias ya existentes en la \ac{apn}en la materia
        \item Diseño y aprobación de una instancia de coordinación para impulsar las labores de planeamiento en la APN
        \item Diseño de instrumentos y sistemas comunes de planificación estratégica y operativa, e implantación de los mismos en al menos 5 organismos o programas de la \ac{apn} adheridos al ámbito de coordinación mencionado anteriormente
        \item Desarrollo de un método para la articulación de la planificación estratégica y operativa con la programación presupuestal, y aplicación de la misma en al menos 5 organismos o programas de la \ac{apn} adheridos al ámbito de coordinación mencionado anteriormente
        \item Desarrollo del módulo de software para la implementación del Plan Anual de Contrataciones (PAC) y su articulación con los procesos de planificación operativa y los sistemas de administración financiera.
        \item Capacitación, difusión e intercambio de experiencias internacionales en temas relativos a la planificación estratégica y operativa. 
    \end{itemize}

  
\paragraph{Componente 2. Fortalecimiento del seguimiento sectorial} \mbox{}\\


Su objetivo es esarrollar una mayor capacidad para el seguimiento de la gestión de gobierno a nivel sectorial. Los productos que se esperan obtener son:

    \begin{itemize}
        \item Desarrollo de un \ac{sigsig} y de sus protocolos de alimentación de datos y control de calidad, como herramienta de seguimiento que se pondrá a disposición de organismos y Proyectos que adhieran al ámbito de coordinación mencionado en el componente anterior.
        \item Apoyo a la implementación del \ac{sigsig} en al menos 5 organismos o programas que formen parte del ámbito de coordinación al que se hace referencia en el componente anterior.
        \item Asistencia técnica para el uso del \ac{sigsig} como herramienta para la toma de decisiones en cada organismo o programa donde se implante.
        \item Capacitación, difusión e intercambio de experiencias internacionales en temas relativos al seguimiento de la gestión 
    \end{itemize}


\paragraph{Componente 3. Promoción de prácticas integrales de gestión por resultados en entidades de la \ac{apn}} \mbox{}\\

Busca fortalecer las prácticas de gestión por resultados en programas y proyectos en entidades de la \ac{apn}. El resultado buscado es la implementación de las prácticas de planificación estratégica y operativa, presupuesto por resultados, procesos administrativos orientados a resultados, etc. en un grupo de organismos o programas de la \ac{apn}, dándose prioridad a los adheridos al ámbito de coordinación mencionado en el Componente 1. Se espera conseguir los siguientes productos:

    \begin{itemize}
        \item Elaboración de documentos marco que establezcan la visión y estrategia de abordaje de la gestión por resultados en cada organismo seleccionado 
        \item Fortalecimiento técnico de las áreas de planeamiento, seguimiento y presupuesto de organismos seleccionados 
        \item Fortalecimiento de otros sistemas de gestión que contribuyen con la \ac{gpr} en organismos seleccionados: compras y contrataciones, administración financiera, etc
        \item Capacitación, difusión e intercambio de experiencias internacionales en temas relativos a la gestión por resultados 
    \end{itemize}

\newpage
\section{Los Sistemas de Monitoreo, Seguimiento y Evaluación en la Administración Pública Nacional}\label{sistemas}

La implementación de la denominada Reforma del Estado en los años '90 introduce nuevos instrumentos para la gestión del Estado. Dentro del conjunto de instrumentos promovidos por dicha Reforma se destaca la configuración de los sistemas de información monitoreo y evaluación. Siendo que este es el objeto del presente documento, no se hace mención a otras innovaciones relevantes como las referidas a las instituciones, instrumentos y técnicas de presupuestación, las de control, u otras específicas de la gestión. También, a efectos de concentrarse en programas o herramientas transversales, se omiten en este listado aquellas experiencias sectoriales, como puede ser el caso de la \ac{diniece} en el Ministerio de Educación o las áreas de evaluación de los programas de financiamiento externo del Ministerio de Salud, por mencionar sólo algunos.

Al momento de comenzar el proceso de Reforma del Estado, la Administración Publica en la Argentina comienza a diseñar e implementar herramientas de producción de información que irán conformando una orientación en la formulación de políticas en las cuáles la información se constituye en un insumo fundamental. Así, los análisis de viabilidad y factibilidad van recogiendo información relativa a los destinatarios de las políticas, a los recursos que moviliza, y gradualmente se van incorporando los dispositivos de monitoreo y evaluación en el diseño de las políticas. 

Claro que los instrumentos generados en esos años tenían la impronta de las recomendaciones de los organismos de financiamiento internacional que procuraban unificar la utilización de sus propias tecnologías de gestión, y la incorporación de sistemas de evaluación y monitoreo en los programas financiados.

Un conjunto de instrumentos diseñados al calor de las orientaciones de la \ac{gpr} señalan la convergencia de esfuerzos institucionales que generan condiciones de viabilidad para la instalación de herramientas como las que nos ocupan. Las experiencias de desarrollo del \ac{siempro}, el \ac{sintys} y el \ac{ssg}, y algunas recientes incorporaciones, como las \ac{ome} el \ac{pepp}, se corresponden con distintos momentos de la reforma del Estado, y actualmente conforman un conjunto de experiencias de implementación de sistemas de información, evaluación y monitoreo,


\subsection{\acrfull{ssg}}

%SIG-SISEG (Sistema Integral de Seguimiento y Evaluación de la Gestión).

Fue diseñado en el marco del Proyecto de Modernización del Estado, creado por \textcite{decreto103}, a los fines de optimizar la gestión y articulación de las distintas áreas de gobierno. Posibilita el seguimiento  y la evaluación de los programas sustantivos de las áreas prioritarias de gobierno contribuyendo a la toma de decisiones para sustentar la coordinación estratégica inter e intraministerial, constituyendo un valioso aporte al fortalecimiento institucional de la \ac{jgm}.

El \ac{ssg} es un sistema de seguimiento de planes, programas y proyectos, con capacidad para sistematizar información y facilitar un monitoreo sistemático de la gestión por parte de los equipos decisores. 

%Si el \ac{sig} como sistema es un desarrollo especializado y potente para el seguimiento, el SISEG es más que el sistema, es el equipo técnico encargado del desarrollo y de la capacitación, entrenamiento y asistencia técnica a los organismos que implementan el sistema. 

Metodológicamente, el sistema de monitoreo parte de la construcción de las matrices de problemas, objetivos y acciones por parte del propio organismo. Estas matrices son las que se despliegan luego en una plataforma informática denominada Tablero de Comando.

En la construcción de cada una de estas matrices es posible la construcción de indicadores específicos de resultados (en relación con los objetivos específicos), de producto y cobertura  (para la acciones) y de ejecución presupuestaria.

El tablero de comando es la herramienta que permite el acceso a los reportes de situación a partir de la interrelación entre las matrices de datos que contiene y que condensan información sobre los problemas, los objetivos y las acciones.

La alimentación del sistema a través de un módulo de carga la realizan los equipos, los usuarios de carga; mientras que el uso mediante consultas está desarrollado en un módulo de consulta que permite observar el desempeño de los indicadores según la  información incorporada.

El Tablero de Comando expresa en su módulo de consulta un sistema de tendencias (medida en que los valores obtenidos crecen, decrecen o se mantienen sin cambio) y un sistema de semáforos que alertan sobre el alcance ("aceptable", "problemático", "deficiente" o "grave") de la meta propuesta.

El \ac{ssg} ha desarrollado también un módulo presupuestario capaz de implementar el seguimiento de un presupuesto construido mediante la metodología de presupuesto por programa.

Desde el año 2014, el equipo del \ac{ssg} se vinculó estrechamente con las \ac{ome}, una nueva herramienta de monitoreo, focalizado en los objetivos y metas de las jurisdicciones.

%\subsubsection*{Situación actual}

%La diseminación de la metodología  de trabajo con el SIG y la aplicación de sus herramientas ha avanzado en dirección a una aplicación acotada a los organismos y/o jurisdicciones que lo demandan, que ha limitado su  pretensión de implementación en toda la Administración Pública Nacional. 

%La aplicación del SIG requiere una fuerte tarea de sensibilización (a nivel de los decisores), capacitación y asistencia técnica a nivel de los ejecutores, de allí que los avances en su implementación remiten a los avances en las etapas de sensibilización (en mayor escala) y de capacitación y asistencia técnica en la preparación de distintas de la metodología.

%El efecto demostración a partir de la utilización del SIG en la Secretaría de Gabinete y en el Ministerio de Trabajo, ha permitido atraer el interés de otros organismos tales como el INDEC, la Subsecretaría de Asuntos Técnico Militares del Ministerio de Defensa, el Programa de Fortalecimiento Institucional del Honorable Senado de la Nación, y la Subsecretaría de Emergencias del Gobierno de la Ciudad Autónoma de Buenos Aires.

%El programa presenta como experiencias más significativas algunas de las siguientes:

%    \begin{itemize}
%        \item La experiencia realizada, con el Ministerio de Trabajo, Empleo y Seguridad Social iniciada en 2004 y que permitió la formulación del Plan Estratégico 2005 -2007 y la aplicación del SIG en su monitoreo. Para 2008 Se elaboró el Plan Estratégico 2008 – 2011, que permitió el desarrollo de ajustes y revisiones permanentes del plan, los indicadores, la metodología y las herramientas.
%        \item Secretaría de Gabinete y Gestión Pública de la Jefatura de Gabinete de Ministros. La misma permitió la elaboración de un Plan estratégico para una Gestión  Pública de Calidad (208 – 2010)
%        \item Subsecretaría de Emergencias del Gobierno de la Ciudad Autónoma de Buenos Aires. En 2006 y 2007 se trabajó en la aplicación de la metodología de Matriz de problemas, objetivos y acciones.
%    \end{itemize}

%Al momento de la entrevista con los Equipos del SISEG se comentó la idea de formulación de la modalidad de implementación del SIG, que permitiera pasar de la sensibilización mediante un módulo de demostración a una utilización más intensiva por las unidades que demandaran el sistema. Las condiciones para el acceso a la metodología y a la tecnología, si avanzara esta perspectiva, así como el uso de la información producida serían parte de nuevos acuerdos con las jurisdicciones.

\subsection{\acrlong{ome}}

A fines del año 2013, la gestión de la \ac{jgm} intentó profundizar el proceso de monitoreo de las actividades de la \ac{apn}, y creó lo que se denominó \acrfull{ome}. Con un fuerte impulso político, la \ac{jgm}, en conjunto con los diferentes ministerios de la \ac{apn} consolidó objetivos y metas planteados por cada uno de ellos para el año 2014. El proceso comenzó sin soporte informático, pero durante el año 2014 se desarrolló un sistema de de carga descentralizado con un módulo de explotación de datos que permite una clara y rápida visualización de la información.

El proceso de definición de metas tiene los siguientes objetivos:
    \begin{enumerate}
        \item Definir y consolidar objetivos estratégicos de política pública y metas de resultado
        \item Contribuir a la instalación de criterios comunes de planificación y programación en torno a resultados de gestión.
        \item Producir información sistemática y confiable sobre el estado de avance en el cumplimiento de \ac{ome} sustantivos de gestión.
    \end{enumerate}
    
Esta iniciativa, superadora del \ac{ssg} en el alcance jurisdiccional, busca crear un entorno de coordinación interministerial, para sistematizar de manera anual, los propósitos institucionales de las dependencias de las \ac{apn}. A partir de la definición de estas metas, se crea un dispositivo de información sobre la planificación, que permite informar periódicamente al Jefe de Gabinete y a las distintas autoridades, sobre el estado de situación de las acciones, de forma trimestral.

Las \ac{ome} incluyeron, en el año 2014, objetivos y metas de 13 ministerios y 4 organismos descentralizados (\acrshort{afip}, \acrshort{anses}, \acrshort{incaa}, \acrshort{sedronar}).

Esta nueva instancia de recolección de metas y objetivos se profundizó en el 2015, y se espera continúe en el 2016.%, teniendo año a año perfecciones metodológicas que ponen en valor el sistema creado por \ac{jgm}.

\subsection{\acrfull{sintys}}

El \ac{sintys} (Sistema de Identificación Nacional Tributario y Social) se crea en 1998 por el Decreto 812, en el ámbito de la Jefatura de Gabinete de Ministros en 1998 y financiado a través de un convenio de préstamo con el \ac{bm}, fue ratificado por la ley 25.345, conocida como ``ley antievasión''. Por su parte, el Pacto Fiscal, aprobado por la ley 25.400, contó con el compromiso de todos los gobernadores de asistir al \ac{sintys} y de crear dispositivos similares al \ac{sintys} en cada una de las provincias. 

El Programa ha evolucionado en dirección a la producción de información social y patrimonial de las personas, y cobra relevancia como sistema de provisión de información, a partir de procedimientos altamente calificados para coordinar  el intercambio de información al interior del Estado. %El \ac{sintys} se define como una red de interconexión de datos que tiene como objetivo comunicar a todas las bases de datos que existen en el país, tanto a nivel municipal, provincial como nacional. 

El \ac{sintys} tiene por misión contribuir a la mayor eficacia y eficiencia de la inversión social y al mejor cumplimiento tributario, coordinando el intercambio de información en función de sus atribuciones legales y de los acuerdos suscriptos con las provincias, los municipios y los organismos nacionales.

El organismo se define como una red de interconexión de datos que tiene como objetivo comunicar a todas las bases de datos que existen en el país, tanto a nivel municipal, provincial como nacional. 

Procura atender algunos de los siguientes problemas relativos a la información  con que cuenta el sector público, entre ellos: la desarticulación entre jurisdicciones, su fragmentación y escasez para la toma de decisiones, y la aplicación  de  criterios disímiles de tratamiento

En este sentido, el \ac{sintys}  se presenta como una herramienta de gestión eficiente que, para el  sector público:
    \begin{itemize}
        \item Facilita la identificación unívoca y homogénea de las personas y la adopción de estándares de intercambio de información gubernamental.
        \item Permite el acceso a los atributos sociales y fiscales mediante la coordinación de Información relacionada con jubilaciones y pensiones y de programas sociales, cobertura de salud, educación, empleo, información patrimonial e impositiva.
        \item Proporciona sólo aquella información que sea de incumbencia del organismo solicitante,
    \end{itemize}

Los usuarios del \ac{sintys} son todos aquellos organismos nacionales, provinciales y municipales que adhieren al Sistema mediante diferentes tipos de acuerdos. Todos ellos acceden a los siguientes servicios derivados del intercambio institucionalizado de información: 
    \begin{itemize}
        \item Validación de datos de personas. 
        \item Identificación del CUIT/CUIL/CDI de las personas. 
        \item Control de supervivencia. 
        \item Elegibilidad para beneficios sociales. 
        \item Detección de incompatibilidades y pluricobertura en la percepción de beneficios sociales, previsionales, de salud y vivienda. 
        \item Verificación de situación laboral (empleo público y privado). 
        \item Detección de posible incumplimiento tributario (impuestos nacionales, provinciales y municipales). 
        \item Comparación de los atributos de identidad de las personas registrados en las bases provinciales con los de padrones nacionales. 
    \end{itemize}

Además, el \ac{sintys} lleva adelante proyectos especiales orientados a la actualización e informatización de registros públicos, en temáticas estratégicas tales como: Registros de Personas Jurídicas Provinciales; Registros de la Propiedad Inmueble Provinciales; Registros Civiles Provinciales; y Generación de \ac{lua}.

En el año 2002, el programa es transferido de la \ac{jgm} al \ac{cncps}, dependiente de la Presidencia de la Nación, y por medio del \textcite{decreto78} se convierte en Dirección Nacional del Sistema de Identificación Nacional Tributario y Social



%\subsubsection*{Situación actual}

El \ac{sintys} presenta hoy, en el conjunto de sistemas de información, un desarrollo ponderado y basado en avanzados criterios de protección de los datos y de tecnología apropiada al cumplimiento de su misión.

Los principales usuarios del sistema son los programas sociales. La valiosa información social y patrimonial de las personas que puede cruzar el sistema se convierte en una herramienta moderna para el seguimiento de la asignación de beneficios o la percepción de derechos, de allí que su uso sea extendido y confiable en cuanto a la información que permite a los programas la validación de sus asignaciones.

Al contener información de grandes bases de datos como la de \ac{afip}, \ac{anses}, \ac{renaper}, registros de propiedad inmueble provinciales, etc., el \ac{sintys} presta al Estado un servicio importante en términos de facilitar el acceso a datos de las personas, particularmente información social y patrimonial. 

\subsection{\acrfull{siempro}}

Creado en el año 1995 por \citeauthor{resolucion2851} de la ex Secretaría de Desarrollo Social,  surge con las características de las herramientas de producción de información que se generaban a partir de las necesidades de los programas financiados por los organismos internacionales de crédito; entre ellas la producción de información para el seguimiento de los programas financiados por estos organismos. En sus modalidades y con aquella impronta, expresa desde sus inicios una fuerte presencia institucional y por lo tanto resulta relevante por su contribución a los diseños de instancias de monitoreo y evaluación de los programas y proyectos focalizadas o de combate a la pobreza. 

En el contexto del surgimiento de estas iniciativas, se encuentran consideraciones relativas a la preocupación por el desarrollo de políticas sociales focalizadas, para lo cual era imprescindible contar con información acerca de las condiciones de vida de la población y, en particular, de la población más carenciada. Junto con el \ac{indec} se desarrolló un Sistema de Información Social para la, entonces, Secretaría de Desarrollo Social y en 1998, comenzó a instrumentarse el \ac{sisfam}, para la construcción de un padrón único de beneficiarios de programas sociales y llevar a cabo un censo de beneficiarios potenciales de programas sociales.

El \ac{siempro} tiene como objetivos: 

    \begin{itemize}
        \item Establecer un sistema de información, evaluación y monitoreo de los programas sociales nacionales.
        \item Desarrollar e implementar el Sistema de Identificación y Selección de Familias Beneficiarias de Programas y Servicios Sociales (SISFAM).
        \item Fortalecer a las áreas sociales nacionales y provinciales en el desarrollo e instalación de sistemas de monitoreo y en la realización de evaluaciones.
        \item Producir nueva información a través de la Encuesta de Desarrollo Social, Condiciones de Vida y Acceso a Programas y Servicios Sociales.
        \item Asegurar la disponibilidad de la información necesaria sobre la población en situación de pobreza y vulnerabilidad social y la ejecución de los programas sociales dirigidos a atenderla.
        \item Capacitar a funcionarios y técnicos de las áreas nacionales y provinciales en política y gerencia social.
        \item Diseminar y transferir metodologías y sistemas de información a las agencias nacionales y provinciales.
        \item Fomentar la vinculación entre el sector académico y el Estado en la investigación y desarrollo de políticas y programas sociales.
    \end{itemize}

Por otro lado, a partir de 1998 comenzó a instrumentarse \ac{sisfam}, siendo coordinado en conjunto con la estrucura del \ac{siempro}. Las líneas de acción del \ac{sisfam} son:
    \begin{itemize}
        \item Construcción del Padrón Único de Beneficiarios de los Programas Sociales.
        \item Censo de Beneficiarios Potenciales de Programas y Servicios Sociales.
    \end{itemize}
% Actualmente, las actividades del Área de Información Social, se centran en la construcción, sistematización y análisis de  indicadores específicos de demanda social. Es, además,  productor de información a partir de datos propios del \ac{siempro}, generados en herramientas como las encuestas y relevamientos realizados por el SISFAM, el seguimiento de indicadores a partir de la utilización de información estadística del SEN, y de los operativos propios para la recolección de información que demandan las políticas de inclusión. 

En términos de estructura, el \ac{siempro}, partir de diciembre de 1999,  pasa a depender de la Secretaría de Tercera Edad y Acción Social, la cual pertenece al Ministerio de Desarrollo Social y Medio Ambiente. Para ese momento, el Gabinete Social funciona en el ámbito de la \ac{jgm}.

Finalmente, en 2002 y mediante el \citeauthor{decreto357} pasa a depender del \ac{cncps} y a partir de 2007 mediante el \citeauthor{decreto78}, obtuvo el rango de Dirección Nacional de Sistemas de Información, Monitoreo y Evaluación de Programas Sociales, pasando el \ac{sisfam} a formar parte de esta Dirección Nacional. 

Actualmente está conformado por 4 áreas: \ac{sisfam}, Monitoreo de Programas Sociales, Evaluación de Programas Sociales, Análisis e Información Social e Informática.

%\subsubsection*{Situación actual}

Desde 2006, el \ac{siempro} ha replanteado sus actividades en función de la nueva realidad del país y de las políticas sociales. El diseño institucional del \ac{siempro} en su calidad de programa con financiamiento externo, contemplaba una estructura de áreas de trabajo conducidas por la figura de gerentes (monitoreo y evaluación, información social e informática) y un cuerpo de consultores especialistas en cada una de esas áreas. Se ha privilegiado, a partir de 2006,  la consolidación de los equipos propios en la temática de la evaluación, diferenciando esta propuesta de las orientaciones que traía el organismo desde sus orígenes, marcados por el suministro de estos servicios mediante contratación externa.

Los bienes y servicios que ofrece son:
    \begin{itemize}
        \item Georeferenciamiento de información social.
        \item Relevamiento de la población beneficiaria en situación de pobreza y vulnerabilidad social.
        \item Evaluación y monitoreo de programas sociales.
        \item Encuesta sobre condiciones de vida.
        \item Informes de situación social y pobreza.
        \item Bases de datos de indicadores sociales.
        \item Base de datos de programas sociales, nacionales y provinciales.
        \item Registros de beneficiarios.
        \item Sistema de identificación y selección de familias beneficiarias actuales y potenciales de programas sociales.
        \item Asistencia técnica y capacitación.
    \end{itemize}
    
Como insumo de los productos ofrecidos, el \ac{siempro} ha desarrollado herramientas y sistemas para el Monitoreo y la Evaluación. 

Respecto al Monitoreo, se ha desarrollado una herramienta de monitoreo denominada \ac{sim}. En sus comienzos era un software descentralizado en cada una de las jurisdicciones que requería la agregación manual de datos. Posteriormente, y a partir del 2008, ha evolucionado a un entorno centralizado, accesible desde un navegador web estándar que permite a los programas la carga y consulta \emph{on line} de la información. Las formas de articulación institucional con los programas y de integración de la información de monitoreo, se guía por las prioridades que establece el \ac{cncps}. Así, el sistema recoge información sobre la ejecución del programa para un periodo determinado y con distintos niveles de localización de la ejecución del programa (municipios, departamentos y provincias). La información sobre metas del programa permite constatar el desempeño del mismo en un periodo determinado de ejecución. Entre los productos originados en el \ac{sim}, el \ac{siempro} brinda a las autoridades de los ministerios y organismos que integran el \ac{cncps}, reportes con información provincial o local, con el respectivo geo-referenciamiento, de los montos ejecutados por programas, el total de prestaciones realizadas en un determinado período y la cobertura de las mismas expresada en cantidad de beneficiarios. En el Apéndice \ref{apendice-siempro} se presenta en detalle los programas que, hasta el año 2013, reportaban al \ac{sim}
En relación a la Evaluación, desde sus orígenes, el \ac{siempro} ha construido una valiosa experiencia en el campo de la evaluación de programas y proyectos, en la que están contenido un conjunto de evaluaciones de programas, en  gran parte con financiamiento internacional, realizadas mediante equipos externos contratados a tal efecto. Muchas de estas evaluaciones, realizadas por Universidades Nacionales, se listan en el Apéndice \ref{apendice-siempro}. 

En la actualidad, el \ac{sim} recibe información de los siguientes ministerios que ejecutan políticas sociales: Ministerio de Desarrollo Social, Ministerio de Planificación Federal, Ministerio de Trabajo, Ministerio de Agricultura. Alrededor de 50 programas sociales nacionales, proveen de información sobre las prestaciones, la cobertura, y los recursos movilizados (detallados en el Apéndice \ref{apendice-siempro}). Por otro lado, esta información no se encuentra disponible en su página web. Los documentos resultantes de las evaluaciones no están disponibles. La información social elaborada por el organismo puede ser solicitada por otros organismos públicos así como también por las instituciones que integran el \ac{cncps}, entre ellos su Consejo Asesor integrado por Organizaciones No Gubernamentales.

\subsection{\acrlong{pepp}}

Con el objetivo de institucionalizar los procesos de evaluación de políticas públicas en la Administración Pública Nacional y potenciar las capacidades para su desarrollo, en el año 2013 la Jefatura de Gabinete de Ministros crea (por Resolución Nº 416/2013)  el \acrfull{pepp}.
Este programa se propone promover la sensibilización, la consolidación en agenda e institucionalizar la evaluación de las políticas públicas en la administración pública nacional; fomentar la investigación aplicada, comparada y participativa. También pretende diseñar metodologías y herramientas de evaluación de políticas públicas susceptibles de ser aplicadas en los organismos gubernamentales; desarrollar capacidades para el diseño e implementación de diversos tipos de evaluación de programas y proyectos; evaluar programas, proyectos y/o políticas implementadas en el ámbito de la administración pública nacional de manera conjunta y en coordinación con los organismos que ejecutan dichas intervenciones, y/o asistir técnicamente para su desarrollo.
Para el cumplimiento de sus objetivos desarrolla acciones en torno a cuatro ejes centrales:

    \begin{itemize}
        \item desarrolla procesos concretos de evaluación de políticas públicas nacionales; 
        \item desarrolla capacidades en materia de evaluación en la Administración Pública Nacional; 
        \item establece directrices y generar conocimientos mediante investigación aplicada en evaluación de políticas públicas y; 
        \item promueve la sensibilización, la consolidación en agenda e institucionalización de la evaluación de políticas públicas en la Administración Pública Nacional.
    \end{itemize}

El programa se encuentra ubicado bajo la órbita de la Secretaría de Gabinete y Coordinación Administrativa pero, debido a su experiencia en procesos vinculados al monitoreo y la evaluación de programas, proyectos y políticas públicas, la conducción operativa está en manos de una Unidad Ejecutora integrada por las Subsecretarías de Gestión y Empleo Público, la Subsecretaría de Evaluación del Presupuesto Nacional y la Subsecretaría de Evaluación de Proyectos con Financiamiento Externo.



%La pertienencia del programa en la estructura de \ac{jgm} tiene sustento en Ley de Ministerios (Ley 22.520), que le otorga entre otras atribuciones, capacidad para:
%    \begin{itemize}
%        \item Coordinar y controlar las actividades de los Ministerios y de las distintas áreas a su cargo, realizando su programación y control estratégico, a fin de obtener coherencia en el accionar de la administración e incrementar su eficacia.
%        \item Entender en la difusión de la actividad del Poder Ejecutivo Nacional, como así también la difusión de los actos del Estado Nacional a fin de proyectar la imagen del país en el ámbito interno y externo.
%    \end{itemize}

Desde 2013 el Programa ha dictado capacitaciones para el personal de la \ac{apn}, ha realizado el "Seminario Internacional de Evaluación", con la presencia de expertos internacionales,  publicó en el sitio web de \ac{jgm} el Banco de evaluaciones de Políticas Públicas, financió de 6 Evaluaciones en diversos organismos y se ha redactado el “Manual de base para la Evaluación de Políticas Públicas”, consensuado con las mas importantes áreas de evaluación de diferentes organismos.



\newpage
\section{Articulación los sistemas de información, evaluación y monitoreo} \label{articulacion}

A lo largo de la Sección \ref{sistemas} se ha detallado el funcionamiento de los principales sistemas de información, monitoreo y evaluación generados a partir de la Reforma del Estado. Estos organismos, a pesar de no formar parte de un sistema formalmente creado, son partes fundamentales en la instauración de la \ac{gpr} en la \ac{apn}

Por este motivo, la \ac{apn} cuenta con valiosas herramientas para desarrollar y implementar un política de \ac{gpr} basadas en las fortalezas de estos sistemas. La fuerte implantación institucional y los desarrollos tecnológicos existentes son dos ventajas con las que cuenta la \ac{apn} a la hora de analizar los pasos a seguir. Por otro lado, es conveniente potencias algunos aspectos para que la \ac{gpr} se transforme un ciclo armonioso, fundamentalmente aquellos referidos a la vinculación del plan-presupeusto.

\subsection{Fortalezas de los sistemas de información, evaluación y monitoreo}
    \begin{itemize}
        \item La implantación institucional de los sistemas: La ubicación de los sistemas de información, evaluación y monitoreo en áreas estratégicas para la toma de decisiones de política pública, resulta un aspectos destacado y por lo tanto una fortaleza de las experiencias y desarrollos institucionales en este campo. En el campo de las políticas sociales, la concentración de gran parte de la función de monitoreo y evaluación en el \ac{cncps} inviste formalmente a estos instrumentos de una funcionalidad relevante, dado que es el ámbito en el que se diseñan las políticas sociales. En relación con la capacidad de monitoreo y evaluación de la gestión, la inserción de esta función en el ámbito de la Secretaría de Gabinete, en tanto colaboradora directa del Jefe de Gabinete, se visualiza como coherente y formalmente potente para proveer de información sobre la marcha de las políticas planes y proyectos. La situación del \ac{pepp} es similar, ya que su pertenencia a la \ac{jgm} le otorga la capacidad de articulación entre los actores de la temática. La capacidad de coordinación e integración de los sistemas de información, evaluación y monitoreo, estaría en condiciones de ser profundizada. Los estilos de gestión son, en todo caso, los proveedores de estímulos o frenos a estas potencialidades. Es decir, el valor que la gestión dé a la producción de información sistemática y periódica sobre la marcha de sus acciones servirá de orientación para profundizar en el desarrollo de los sistemas.
        \item Los desarrollos tecnológicos y conceptuales: Siendo que los avances en el campo de las tecnologías de información y transmisión de datos son cada vez más accesibles y forman parte de la exigencias de vinculación al medio, los recursos humanos de los organismos pasan a ser un componente prioritario.  Las capacidades profesionales y técnicas de los organismos involucrados (\ac{siempro}, \ac{sintys}, \ac{ssg}, etc.) facilitan la introducción de innovaciones y perspectivas relativas a la mejora de la gestión de estos sistemas. Hay una notable experiencia acumulada en la producción de información para el monitoreo y la evaluación en el propio sector público.
    \end{itemize}

\subsection{Aspectos a mejorar en los sistemas de información, evaluación y monitoreo}

    \begin{itemize}
        \item Integración transversal de los sistemas. La integración transversal de los sistemas de información, evaluación y monitoreo con los respectivos procesos planificación, presupuesto y control son incipientes y aún insuficientes.
        \item En relación con el presupuesto. En todos los casos, es notoria la falta de articulación con el presupuesto nacional y con las áreas que producen esta información.  Los déficit y dificultades remiten a aspectos de integración y coordinación que dificultan el acceso a la información actualizada de la ejecución presupuestaria; otras, a dificultades técnicas relativas a la necesidad de compatibilización de formatos, dimensiones y elementos propios del tratamiento y procesamiento de la información presupuestaria y su homologación con los sistemas de monitoreo y evaluación. 
        \item En relación con el planeamiento. La articulación a este nivel en el mejor de los casos se da a nivel de programas o proyectos. En el caso del SIG, el sistema se origina idealmente en el plan estratégico de un organismo o un plan o un proyecto. En el caso del SIEMPRO, el sistema de monitoreo permite desagregar aspectos (prestaciones, unidades de medida, beneficiarios, presupuestos) y tomar las metas presupuestarias para verificar los avances, si bien esta actividad  no está directamente relacionado con el momento de planificación de las políticas, programas  y/o proyectos.
    \end{itemize}

\subsection{Funcionamiento de los sistemas}
    \begin{itemize}
        \item La provisión de información: la alimentación de los sistemas tienen las dificultades propias de una articulación débil entre programas aún de un mismo ministerio. En el caso del \ac{siempro}, una buena cantidad de programas y proyectos proveen de información de la ejecución, si bien en muchos casos, dicha provisión no es regular. En el caso del \ac{ssg}, actualmente, es pequeño el número de programas el que utiliza el sistema. La información producida, en relación con el desempeño del programa que lo está aplicando, es importante para el propio organismo productor de la información. Al ser acotado el universo de programas y jurisdicciones que lo utilizan, su contribución al seguimiento estratégico como el que se propone en sus objetivos, es escasa aún. 
        \item Los usuarios: En el caso del \ac{siempro}, es el propio \ac{cncps} el que utiliza la información, los ministerios, organismos, instituciones que lo integran son sus principales usuarios.  Habitualmente los ministerios, en particular el Ministerio de Desarrollo Social, solicitan reportes de ejecución de planes y/o proyectos en algunas áreas del país. El organismo puede responder ágilmente a dichas consultas. Otros usos, en particular los accesos públicos a información no están disponibles. En el caso del \ac{ssg}  el principal usuario es el propio programa productor de a información, o sus áreas jerárquicas de dependencia. En el caso del \ac{pepp}, los usuarios son programas u organismos que voluntariamente se ofrecen a ser evaluados, pero con resultados optimistas para el futuro dado su reciente creación.
    \end{itemize}



\newpage
\section{Conclusiones} \label{conclusiones}

Los sistemas descriptos en el documento representan un conjunto de instrumentos que configuran la función de monitoreo y evaluación del Estado. Por lo tanto, existe en la Administración Pública Nacional una base normativa que expresa las orientaciones que fueron dándose bajo las ideas de gestión "hacia adentro del Estado" surgidas en los '90, y ampliadas desde los primeros años de este siglo. Los diversos instrumentos originados en esos años, muchos de ellos con importante base normativa que respalda su accionar, se han ido agregando en el sistema institucional con distinto grado de articulación.

En el campo de las políticas sociales, la concentración de gran parte de la función de monitoreo y evaluación en el \ac{cncps} inviste formalmente a estos instrumentos de una funcionalidad relevante, dado que es el ámbito en el que se diseñan las políticas sociales. 

En relación con la capacidad de monitoreo y evaluación de la gestión, la inserción de esta función en el ámbito de la Secretaría de Gabinete y Gestión Pública, en tanto colaboradora directa del Jefe de Gabinete de Ministros, se visualiza como coherente y formalmente potente para proveer de información sobre la marcha de las políticas planes y proyectos.
La capacidad de coordinación e integración de los sistemas de información, evaluación y monitoreo, estaría en condiciones de ser profundizada. Los estilos de gestión son, en todo caso, los proveedores de estímulos o frenos a estas potencialidades. Es decir, el valor que la gestión brinda a la producción de información sistemática y periódica sobre la marcha de sus acciones servirá de orientación para profundizar en el desarrollo de los sistemas.

Siendo que los avances en el campo de las tecnologías de información y transmisión de datos son cada vez más accesibles y forman parte de las exigencias de vinculación al medio, los recursos humanos de los organismos pasan a ser un componente prioritario. Las capacidades profesionales y técnicas de los organismos involucrados en el desarrollo y gestión de los sistemas de información disponibles (\ac{siempro}, \ac{sintys}, \ac{ssg}, etc.) facilitan la introducción de innovaciones y perspectivas relativas a la mejora de la gestión de estos sistemas. Hay una notable experiencia acumulada en la producción de información para el monitoreo y la evaluación en el propio sector público.

La integración transversal de los sistemas de información, evaluación y monitoreo con los respectivos procesos de planificación, presupuesto y control son incipientes, siendo necesaria la profundización en tal sentido. Al respecto, iniciativas actuales tanto de la \ac{jgm} como de la \ac{sh} a través de la SSP, permiten prever avances de importancia en dichas materias en el corto y mediano plazo.

En todos los casos, aún se observa una significativa falta de articulación entre el presupuesto nacional y otras áreas productoras de información. Los déficit y dificultades remiten a aspectos de integración y coordinación que dificultan el acceso a la información actualizada de la ejecución presupuestaria; otras, a dificultades técnicas relativas a la necesidad de compatibilización de formatos, dimensiones y elementos propios del tratamiento y procesamiento de la información presupuestaria y su homologación con los sistemas de monitoreo y evaluación. 

La articulación a este nivel en el mejor de los casos se da a nivel de programas o proyectos. En el caso del \ac{ssg}, el sistema se origina idealmente en el plan estratégico de un organismo o un plan o un proyecto. En el caso del \ac{siempro}, el sistema de monitoreo permite desagregar aspectos (prestaciones, unidades de medida, beneficiarios, presupuestos) y tomar las metas presupuestarias para verificar los avances, si bien esta actividad no está directamente relacionado con el momento de planificación de las políticas, programas y/o proyectos.

La interacción del \ac{siempro} con diferentes programas, tanto nacionales como provinciales, hace posible el desarrollo potenciado de información geográfica. En el caso del \ac{ssg}, al ser acotado el universo de programas y jurisdicciones que lo utilizan, su contribución al seguimiento estratégico como el que se propone en sus objetivos, es aún limitada. En el caso del \ac{siempro}, es el propio \ac{cncps} el que utiliza la información, es decir, los ministerios, organismos e instituciones que lo integran son sus principales usuarios. En el caso del \ac{ssg} el principal usuario es el propio programa productor de la información, o sus áreas jerárquicas de dependencia.

En la \ac{apn} han existido experiencias heterogéneas en cuanto a la evaluación de las políticas públicas, con diferentes niveles de éxito. La existencia de diversas experiencias de evaluación, aún desarrolladas en forma aislada y sin inserción y articulación institucional, siempre fue vista como una posible plataforma para la conformación de un sistema articulado de evaluación de políticas públicas. A partir de este diagnóstico, la cración del \ac{pepp} a puesto a disposición una gran herramienta para coordinar y potenciar los diferentes esfuerzos jurisdiccionales vinculados a la evaluación, como así también para generar una cultura que permita, en el mediano plazo, incorporar a la evalaución en el ciclo de gestión de las políticas públicas. 
\newpage
\appendix
\section{Apendice}\label{apendice-siempro}

\subsection{Programas monitoreados por el \acrshort{siempro}}

A continuación se presenta un listado de programas que reportan al \ac{sim} \footnote{Fuente: \citetitle{siempro2012}}:
    \begin{itemize}
        \item Dirección de Coordinación de Logística - Plan AHI
        \item Pensiones No Contributivas
        \item Programa Familias por la Inclusión Social
        \item ANSES - Pensiones a Ex Combatientes de Malvinas
        \item Programa Nacionales de Empleo (Jefes de Hogar, Seguro de Capacitación y Empleo, Jóvenes y otros)
        \item PAMI - Internación en Residencias para Adultos Mayores
        \item Dirección de Infraestructura Educativa - Programa 37- Infraestrucutra y Equipamiento
        \item Programa de Desarrollo Social en Áreas Fronterizas del Noa y Nea  - PROSOFA
        \item Programa de Mejoramiento Habitacional e Infraestructura Social Basica PROMHIB
        \item Programa de Provisión de Agua Potable, Ayuda Social y Saneamiento Básico - PROPASA
        \item Programa de Reactivación para las Obras del Fondo Nacional de la Vivienda - (FONAVI) - Reactivación II
        \item Programa Federal de Construcción de Viviendas
        \item Programa Federal de Emergencia Habitacional
        \item Programa Federal de Mejoramiento de Viviendas Mejor Vivir
        \item Programa Nacional 700 Escuelas
        \item Programa Federal de Solidaridad Habitacional
        \item Dirección de Asistencia Directa a Personas por Situaciones Especiales
        \item PAMI - Programa para el Bienestar de los Mayores - Probienestar 
        \item PAMI - Subsidios de Asistencia Socio-sanitaria
        \item Refuerzo a los Servicios Alimentarios Escolares
        \item PNSA - Financiamiento de proyectos productivos, capacitación y fortalecimiento institucional
        \item Centros Integradores Comunitarios
        \item ANSES - Prestaciones por Desempleo
        \item Coordinación de Asistencia Directa a Instituciones
        \item Dirección Nacional de la Juventud
        \item PNSA - Asistencia Alimentaria Directa - Federal
        \item PNSA- Asistencia Alimentaria Directa - Focalizado
        \item PNSA - Asistencia alimentaria directa - Compra centralizada
        \item PNSA - Asistencia Alimentaria Directa -  Plan Ahí
        \item PNUD - ARG
        \item Dirección de Coordinación de Logística - Ayudas Urgentes
        \item Dirección de Coordinación de Logística - Talleres Familiares 
        \item Dirección Nacional de Apoyo Logístico
        \item Instituto Nacional de Asuntos Indígenas - INAI
        \item Cambio Rural - Programa Federal de Reconversión Productiva para Pequeñas y Medianas Empresas Agropecuarias 
        \item MINIFUNDIO -  Programa de Investigación y Extensión para los Productores Minifundistas
        \item PROFAM - Programa para Productores Familiares
        \item Plan Manos a la Obra - Emprendimientos Productivos
        \item Fondo de Capital Social
        \item Programa Integral para la Igualdad Educativa - PIIE
        \item Comisión Nacional Asesora para la Integración de Personas Discapacitadas - CONADIS
        \item Programa de Reconversión de Áreas Tabacaleras - PRAT
        \item Programa Federal de Salud - PROFE ``SALUD''
        \item Programa Social Agropecuario - PSA
        \item Programa Comer en Familia
        \item Pequeños Ganaderos
    \end{itemize}


\subsection{Evaluaciones realizadas por el \acrshort{siempro}} 

Listado de evaluaciones realizadas por el \acrshort{siempro}, tabuladas por año, organismo o programa evaluado y tipo de evaluación \footnote{Fuente: \citetitle{siempro2012}}:

    \begin{description}
    \item{2007}
        \begin{itemize}
            \item Plan Manos a la Obra - Medio término
            \item PROMEBA – Final
        \end{itemize}
    \item{2006}
        \begin{itemize}
            \item Componente Materiales – Tipología 6 del Plan Jefes y Jefas de Hogar Segunda medición
            \item IDH – Componente I Transferencia de ingresos a las familias - Final
            \item PROMEBA - FINAL
            \item Proyectos financiados por Cooperación Italiana - Diagnóstica
        \end{itemize}
    \item{2005}
        \begin{itemize}
        \item COMER EN FAMILIA Diagnóstica
        \item Proyectos financiados por Cooperación Italiana - Diagnóstica
        \item FOPAR - FINAL
        \item Componente Materiales – Tipología 6 del Plan Jefes y Jefas de Hogar Primera medición
        \end{itemize}
    \item{2004}
        \begin{itemize}
            \item FOPAR Diagnóstica 2da Etapa
            \item PROAME II Medio Término
            \item PROAPS - REMEDIAR - Medio Término
            \item Desarrollo de comunidades indígenas (DCI)- Desde la perspectiva de los beneficiarios 
        \end{itemize}
    \item{2003}
        \begin{itemize}
            \item CENOC Investigación Evaluativa
            \item PROSOFA Medio Término
            \item Programas Alimentarios de la Provincia de Buenos
            \item Aires - Diagnóstica
            \item PROAPS - REMEDIAR - Línea de base
            \item FOPAR Diagnóstica - 1ra Etapa
        \end{itemize}
    \item{2001}
        \begin{itemize}
            \item Concurso de Proyectos Sociales del Fondo de Inversión y Desarrollo Social (FIDES)
de la provincia de Mendoza - Ex post de proyectos
            \item PAGV Componente de Atención a la Población Indígena (CAPI). Medio Término
            \item PAGV Componente Fortalecimiento Institucional (Sistema Único de Identificación y Registro de Familias Beneficiarias de  Programas Sociales – SISFAM). Medio Término
            \item FOPAR II - Línea de Base
        \end{itemize}
        
    \item{2000}
        \begin{itemize}
        \item Programas de Desarrollo Infantil en la provincia de La Pampa. Investigación Evaluativa
            \item PAGV Componente VASS - Medio Término
            \item PAGV Componente Fortalecimiento Institucional. (Sistema Único de Identificación y Registro de Familias Beneficiarias de Programas Sociales – SISFAM) Línea de Base
        \end{itemize}
    \end{description}


% Bibliografía
\newpage
%\addcontentsline{toc}{section}{\protect\numberline{\thesection} Referencias} %Agregamos las referencias al índice, con número
\addcontentsline{toc}{section}{Referencias} %Agregamos las referencias al índice, sin número
\nocite{*} % Listamos todo el archivo bib, no sólo lo citado en el cuerpo
\printbibheading
\printbibliography[keyword={citar-bib},heading=subbibliography,title={Bibliografía}]
\printbibliography[keyword={entrevistas},heading=subbibliography,title={Entrevistas}]
\printbibliography[keyword={normativa-listar},heading=subbibliography,title={Normativa}]
\printbibliography[keyword={sitiosweb},heading=subbibliography,title={Sitios Web}]

\end{document}