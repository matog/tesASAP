\newpage
\section*{Evaluación}

\paragraph{I-  AUTOR DEL TRABAJO FINAL:} \mbox{}\\
Matías Manuel Grandi \\
Licenciado en Economía \\
2012-2013\\
Fecha de presentación: noviembre de 2015
\paragraph{II- TÍTULO DEL TRABAJO:} \mbox{}\\
Los Sistemas de Información, Monitoreo y Evaluación y su vinculación con la Gestión por Resultados. El caso de la Administración Pública Nacional de Argentina
\paragraph{III- COMENTARIOS:} \mbox{}\\
\paragraph{A. Contenido abreviado} \mbox{} \\
Este escrito es un documento cuyo contenido realiza una descripción de los sistemas de monitoreo y evaluación vigentes en la Administración Pública Nacional analizando la potencialidad de la integración de las tareas de los diferentes sistemas para lograr una más efectiva utilización de las herramientas de la evaluación. Ello fundamentado en la importancia de esa tarea para poder obtener mejoras concretas en las acciones de planificación en las Jurisdicciones y Entidades del Gobierno y en la ejecución de las políticas por parte de esos organismos.\\
\paragraph{B. Comentarios específicos:} \mbox{} \\
    \begin{itemize}
        \item Sobre su pertinencia y alcance estimo que el trabajo presenta una síntesis de los principales sistemas de monitoreo y evaluación que abarcan un amplio espectro de esa actividad en la APN.
        \item El análisis es el apropiado a los efectos de sustentar las conclusiones a las que arriba el autor.
        \item El desarrollo es fluido y no hay información redundante y repetida que obstaculice la lectura y la comprensión de los objetivos que se plantea el autor.
        \item Sobre la bibliografía utilizada: Presenta una extensa bibliografía referida a las políticas públicas, la gestión por resultados, y las implementación de los sistemas de monitoreo y evaluación, con un detallado análisis normativo, y entrevistas personales a los encargados de los organismos de mayor trascendencia en la materia.
        \item La organización y escritura del trabajo son adecuadas y concluyen en el buen resultado de un texto que resulta interesante en su desarrollo y sostiene adecuadamente las conclusiones a la que arriba.
        \item El trabajo es muy bueno en la fundamentación del porqué de la necesidad de una integración de las áreas de evaluación de la APN a los efectos de la mejora de la gestión de las políticas públicas por parte de los que planifican y ejecutan las mismas.
    \end{itemize}
\paragraph{IV- NOTA PROPUESTA} \mbox{} \\
Se propone la nota de 8 (ocho) puntos.
 
 
\vspace{3cm}
Lic. Norberto Carlos Perotti

\vspace{0.1cm}
Buenos Aires, 12 de noviembre de 2015
