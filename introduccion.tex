\newpage
\section{Introducción}\label{introduccion}

Los sistemas de información, monitoreo y evaluación del sector público han tomado gran relevancia como herramientas para la toma de decisiones en las ultimas dos décadas. La necesidad de contar con información sobre el rumbo de la gestión y el desempeño de las políticas públicas volvió a centrar la atención de los decisores en estas actividades, lo que provocó el desarrollo de nuevas áreas, programas y organismos para cumplir con esas funciones.

Por otro lado, estos sistemas forman parte de los pilares fundamentales de la \ac{gpr}, que aparecen generalmente citados en la bibliografía sobre el tema. La \ac{gpr} es una herramienta de gestión que se instaló en las agendas de los gobiernos desde mediados de la década de los '80, con el surgimiento de la \ac{ngp}, principalmente en Reino Unido, Australia y Nueva Zelanda. En los países de América Latina, con los procesos de reforma de la década de los '90, se comenzaron a desarrollar instrumentos para mejorar la gestión del Estado ``hacia adentro''. La \ac{gpr} no tiene un modelo unívoco de implementación. Las características propias de la normativa, la cultura organizacional y las formas de gobierno transforman cada burocracia en un caso único.

En Argentina, en la \ac{apn}, han existido diferentes intentos para dotar a la gestión de información de calidad, como para implantar sistemas de evaluación dentro del ciclo de gestión de las políticas públicas. El éxito de cada una de estas instancias ha dependido, en gran parte, del impulso político, la capacidad técnica, y la fortaleza institucional. En el Poder Ejecutivo nacional de Argentina, las capacidades institucionales para diseñar, planificar, implementar, monitorear y evaluar políticas difieren entre ministerios y agencias de gobierno, provocando niveles heterogéneos en la calidad de las intervenciones. Esto no implica que no exista un sistema de \ac{gpr}, pero si que las diferentes instancias de la misma estén con diversos niveles de implementación y funcionamiento.

El presente documento busca indagar sobre el funcionamiento de los sistemas, programas u organismos creados a la luz del procesos iniciado en los '90 con la denominada Reforma del Estado, y su vinculación con la \ac{gpr}. Dentro los nuevos instrumentos generados por dicha Reforma, se destacan los sistemas de información, monitoreo y evaluación de programas y políticas. Por este motivo, no se hará mención de otras importantes innovaciones como las referidas a la técnica presupuestaria, de importante avance en los últimos 20 años.

En la Sección \ref{gpr}, se presenta un definición conceptual de la \ac{gpr},  la cronología de su desarrollo en en la \ac{apn}, con un detallado análisis de la principal normativa vinculada al proceso y la descripción del \ac{prodev}, el programa que buscar apoyar la implementación de la \ac{gpr} en Argentina. En la Sección \ref{sistemas} se describen los principales sistemas de información, monitoreo y evaluación generados en la \ac{apn}, y en la Sección \ref{articulacion} se caracteriza el funcionamiento de los organismos descritos en el punto anterior, focalizándose en las necesidades que la herramienta de la \ac{gpr} demandan de los mismos. Por último, se elabora una conclusión en función del análisis realizado.