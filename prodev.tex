\newpage
\subsection{¿Qué es el PRODEV?} \label{prodev}

El \ac{prodev} es una iniciativa del \ac{bid}, que  abarca una serie de acciones específicas que procuran fortalecer la efectividad de los gobiernos de la región, a fin de que puedan alcanzar mejores resultados en sus intervenciones de desarrollo.
Según el sitio web del \ac{bid} \footnote{\citeurl{bid}}:

\begin{quote}
\small ¿Cuál es el objetivo del \ac{prodev}? El objetivo principal del \ac{prodev} es apoyar a los países miembros prestatarios, interesados en mejorar la gestión del sector público (incluido el diseño, la ejecución, el seguimiento y la evaluación de políticas, estrategias, programas y proyectos), de una forma coherente con la asignación y el uso eficientes de los recursos de los ministerios y departamentos centrales (finanzas, planificación y presupuesto), ministerios sectoriales (salud, infraestructura y educación) y gobiernos subnacionales (estatales, provinciales, municipales y locales).  
\end{quote}

A efectos de institucionalizar el programa, el gobierno argentino crea en 2009 una cooperación técnica no reembolsable y una \ac{ueprodev} en la entonces Secretaría de Evaluación Presupuestaria de la \ac{jgm}, por la cual el \ac{bid} financia la implementación del \ac{prodev} en Argentina. Sus objetivos específicos eran:  
    \begin{itemize}
        \item analizar la situación de las distintas áreas clave para una gestión por resultados y, con base en la información relevada, preparar un diagnóstico y un plan de acción para el país; y 
        \item sensibilizar y capacitar a directivos y técnicos del sector público con el fin de generar el cambio cultural e institucional necesario para la implantación de la gestión pública por resultados. 
    \end{itemize}

Cumpliendo con su segundo objetivo específico, la \ac{ueprodev} se contacta con diversos organismos de la \ac{apn} para plantear una serie de acciones en conjunto con el fin de apoyar la implementación de la \ac{gpr} en los mismos, buscando una mayor eficacia y calidad de las Políticas Públicas.

La \ac{jgm} ejecuta el Programa a través de una unidad ejecutora creada al efecto en la Subsecretaría de Evaluación del Presupuesto Nacional, siendo la Secretaría de Evaluación Presupuestaria de \ac{jgm} la coordinadora del Programa.

Con el fin de facilitar la ejecución, coordinación y sostenibilidad del Programa, se creó formalmente un Comité Asesor integrado por el Subsecretario de Evaluación del Presupuesto Nacional y el Subsecretario de Gestión y Empleo Público de la \ac{jgm} y el Subsecretario de Presupuesto del \ac{mecon}.

A su vez, coordina la acción conjunta con otros actores estratégicos de la \ac{apn}, de acuerdo a cada uno de los los pilares fundamentales que impulsa y promueve la \ac{ueprodev}:

    \begin{enumerate}
        \item En materia de Planificación Estratégica, la \ac{onig} de la Subsecretaría de Gestión y Empleo Público de la Jefatura de Gabinete de Ministros.
        \item en relación al Presupuesto, la oficina \ac{onp} de la Subsecretaría de  Presupuesto del \ac{mecon}
        \item respecto a la evaluación, el Programa de Evaluación de Políticas Públicas, ejecutado conjuntamente entre la Secretaría Evaluación Presupuestaria y la Secretaría de Gabinete y Gestión Administrativa de la \ac{jgm}. 
    \end{enumerate}

La implementación del \ac{prodev} en Argentina fue diseñado para que su ejecución se cumpla en dos etapas. La primera, denominada Cuenta A, se comenzó a ejecutar durante el año 2011, finalizando su ejecución en diciembre 2012. Posteriormente, la Cuenta B fue aprobada en diciembre 2013 (\cite{cartaprodevb}), y actualmente se encuentra en proceso de ejecución. A continuación se detallan, algunas de las principales actividades realizadas por el Programa en pos de apoyar la implementación de la \ac{gpr} según se describen en \citetitle{informegestionprodev} (\citeyear{informegestionprodev}).

\subsubsection{PRODEV Cuenta A: Actividades}

La cuenta A tenía como foco la sensibilización, capacitación y  elaboración de un diagnóstico sobre la \ac{gpr} en el país. Las actividades desarrolladas a tal fin fueron:
    \begin{itemize}
        \item Jornadas ASAP de Capacitación sobre Planeamiento Estratégico (Diciembre 2012)
%        \item Taller de intercambio de experiencias para los funcionarios que participan en la articulación de las acciones de las áreas que intervengan en la asistencia técnica en la implementación de \ac{gpr}. 25 participantes.
        \item Taller de sensibilización en los organismos y jurisdicciones.
        \item Diagnóstico Estado de Situación \ac{gpr} en Argentina. 
        \item Coordinación Misión \ac{sep}: La unidad ejecutora \ac{prodev} coordinó las entrevistas y apoyó el relevamiento del Sistema de Evaluación Prodev 2012 del \ac{bid}
        \item Metodología para una Estrategia de Intervención: Documento que desarrolla una metodología de intervención organizacional coordinada por parte de los pilares fundamentales de la GpRD, articulando la planificación, el presupuesto y la evaluación.
        \item Articulación Planificación Estratégica-Operativa-SISEG: A partir del trabajo entre los directivos y técnicos de las áreas de planificación, monitoreo y la  Unidad Ejecutora del Prodev se consensuaron un conjunto de acuerdos de articulación para el trabajo conjunto a replicarse en organismos de la APN.
        \item \acrshort{incucai}: Se completó exitosamente la formulación del plan estratégico y operativo, y se articuló con el \ac{mecon} para la formulación de indicadores de resultados y revisión de la estructura presupuestario.   
        \item Articulación con MTEySS: dado que el organismo viene desarrollando hace varios años importantes avances en planificación estratégica y operativa, gestionó la cesión del nuevo tablero de control (\ac{ssg}) para el seguimiento de sus indicadores.
        \item Capacitaciones CEPAL y AECID: Técnicos \ac{prodev} realizaron capacitaciones en \ac{cepal} (Chile) sobre “Matriz de Marco Lógico” y en la \ac{aecid} (Santa Cruz de la Sierra) sobre “Evaluación de Políticas Públicas”.
        \item Conferencia Internacional: Participación del Subsecretario en la “Third International Conference on National Evaluation Capacities” (San Pablo, 2013) organizada por \acrshort{pnud} y el Gobierno del Brasil para presentar los avances del \ac{prodev} en Argentina y documentos teóricos de articulación plan-presupuesto-evaluación.
        \item Programa de Evaluación de Políticas Públicas: La Unidad Ejecutora del \ac{prodev} formó parte de la elaboración y la gestión del Programa de Evaluación de Políticas Públicas, junto con la Subsecretaría de Gestión y Empleo Público y la Subsecretaría de Evaluación de Proyectos con Financiamiento Externo. El programa tiene como objetivo principal institucionalizar los procesos de evaluación de políticas públicas en la Administración Pública Nacional y potenciar las capacidades para su desarrollo.
    \end{itemize}

\subsubsection{PRODEV Cuenta B: Actividades}
La cuenta B del \ac{prodev} se encuentra actualmente comenzando su ejecución y tiene planteadas las siguientes actividades, según se explicita en el \citetitle{poamedianoplazo}:
\paragraph{Componente I. Sistema Nacional de Planificación} \mbox{}\\

Tiene como objetivo dotar a la planificación en la \ac{apn} de mayor capacidad estratégica y operativa
Resultados esperados: dotar a la \ac{apn} con una propuesta de marco conceptual, institucional, normativo y metodológico para la elaboración de planes estratégicos y operativos, vinculados con la programación presupuestal; e implantar instrumentos y sistemas de planificación en un conjunto de ministerios de la APN

Los productos esperados son: 
    \begin{itemize}
        \item Diseño conceptual, institucional y normativo del Sistema Nacional de Planificación, recogiendo las experiencias ya existentes en la \ac{apn}en la materia
        \item Diseño y aprobación de una instancia de coordinación para impulsar las labores de planeamiento en la APN
        \item Diseño de instrumentos y sistemas comunes de planificación estratégica y operativa, e implantación de los mismos en al menos 5 organismos o programas de la \ac{apn} adheridos al ámbito de coordinación mencionado anteriormente
        \item Desarrollo de un método para la articulación de la planificación estratégica y operativa con la programación presupuestal, y aplicación de la misma en al menos 5 organismos o programas de la \ac{apn} adheridos al ámbito de coordinación mencionado anteriormente
        \item Desarrollo del módulo de software para la implementación del Plan Anual de Contrataciones (PAC) y su articulación con los procesos de planificación operativa y los sistemas de administración financiera.
        \item Capacitación, difusión e intercambio de experiencias internacionales en temas relativos a la planificación estratégica y operativa. 
    \end{itemize}

  
\paragraph{Componente 2. Fortalecimiento del seguimiento sectorial} \mbox{}\\


Su objetivo es esarrollar una mayor capacidad para el seguimiento de la gestión de gobierno a nivel sectorial. Los productos que se esperan obtener son:

    \begin{itemize}
        \item Desarrollo de un \ac{sigsig} y de sus protocolos de alimentación de datos y control de calidad, como herramienta de seguimiento que se pondrá a disposición de organismos y Proyectos que adhieran al ámbito de coordinación mencionado en el componente anterior.
        \item Apoyo a la implementación del \ac{sigsig} en al menos 5 organismos o programas que formen parte del ámbito de coordinación al que se hace referencia en el componente anterior.
        \item Asistencia técnica para el uso del \ac{sigsig} como herramienta para la toma de decisiones en cada organismo o programa donde se implante.
        \item Capacitación, difusión e intercambio de experiencias internacionales en temas relativos al seguimiento de la gestión 
    \end{itemize}


\paragraph{Componente 3. Promoción de prácticas integrales de gestión por resultados en entidades de la \ac{apn}} \mbox{}\\

Busca fortalecer las prácticas de gestión por resultados en programas y proyectos en entidades de la \ac{apn}. El resultado buscado es la implementación de las prácticas de planificación estratégica y operativa, presupuesto por resultados, procesos administrativos orientados a resultados, etc. en un grupo de organismos o programas de la \ac{apn}, dándose prioridad a los adheridos al ámbito de coordinación mencionado en el Componente 1. Se espera conseguir los siguientes productos:

    \begin{itemize}
        \item Elaboración de documentos marco que establezcan la visión y estrategia de abordaje de la gestión por resultados en cada organismo seleccionado 
        \item Fortalecimiento técnico de las áreas de planeamiento, seguimiento y presupuesto de organismos seleccionados 
        \item Fortalecimiento de otros sistemas de gestión que contribuyen con la \ac{gpr} en organismos seleccionados: compras y contrataciones, administración financiera, etc
        \item Capacitación, difusión e intercambio de experiencias internacionales en temas relativos a la gestión por resultados 
    \end{itemize}
