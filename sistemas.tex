\newpage
\section{Los Sistemas de Monitoreo, Seguimiento y Evaluación en la Administración Pública Nacional}\label{sistemas}

La implementación de la denominada Reforma del Estado en los años '90 introduce nuevos instrumentos para la gestión del Estado. Dentro del conjunto de instrumentos promovidos por dicha Reforma se destaca la configuración de los sistemas de información monitoreo y evaluación. Siendo que este es el objeto del presente documento, no se hace mención a otras innovaciones relevantes como las referidas a las instituciones, instrumentos y técnicas de presupuestación, las de control, u otras específicas de la gestión. También, a efectos de concentrarse en programas o herramientas transversales, se omiten en este listado aquellas experiencias sectoriales, como puede ser el caso de la \ac{diniece} en el Ministerio de Educación o las áreas de evaluación de los programas de financiamiento externo del Ministerio de Salud, por mencionar sólo algunos.

Al momento de comenzar el proceso de Reforma del Estado, la Administración Publica en la Argentina comienza a diseñar e implementar herramientas de producción de información que irán conformando una orientación en la formulación de políticas en las cuáles la información se constituye en un insumo fundamental. Así, los análisis de viabilidad y factibilidad van recogiendo información relativa a los destinatarios de las políticas, a los recursos que moviliza, y gradualmente se van incorporando los dispositivos de monitoreo y evaluación en el diseño de las políticas. 

Claro que los instrumentos generados en esos años tenían la impronta de las recomendaciones de los organismos de financiamiento internacional que procuraban unificar la utilización de sus propias tecnologías de gestión, y la incorporación de sistemas de evaluación y monitoreo en los programas financiados.

Un conjunto de instrumentos diseñados al calor de las orientaciones de la \ac{gpr} señalan la convergencia de esfuerzos institucionales que generan condiciones de viabilidad para la instalación de herramientas como las que nos ocupan. Las experiencias de desarrollo del \ac{siempro}, el \ac{sintys} y el \ac{ssg}, y algunas recientes incorporaciones, como las \ac{ome} el \ac{pepp}, se corresponden con distintos momentos de la reforma del Estado, y actualmente conforman un conjunto de experiencias de implementación de sistemas de información, evaluación y monitoreo,


\subsection{\acrfull{ssg}}

%SIG-SISEG (Sistema Integral de Seguimiento y Evaluación de la Gestión).

Fue diseñado en el marco del Proyecto de Modernización del Estado, creado por \textcite{decreto103}, a los fines de optimizar la gestión y articulación de las distintas áreas de gobierno. Posibilita el seguimiento  y la evaluación de los programas sustantivos de las áreas prioritarias de gobierno contribuyendo a la toma de decisiones para sustentar la coordinación estratégica inter e intraministerial, constituyendo un valioso aporte al fortalecimiento institucional de la \ac{jgm}.

El \ac{ssg} es un sistema de seguimiento de planes, programas y proyectos, con capacidad para sistematizar información y facilitar un monitoreo sistemático de la gestión por parte de los equipos decisores. 

%Si el \ac{sig} como sistema es un desarrollo especializado y potente para el seguimiento, el SISEG es más que el sistema, es el equipo técnico encargado del desarrollo y de la capacitación, entrenamiento y asistencia técnica a los organismos que implementan el sistema. 

Metodológicamente, el sistema de monitoreo parte de la construcción de las matrices de problemas, objetivos y acciones por parte del propio organismo. Estas matrices son las que se despliegan luego en una plataforma informática denominada Tablero de Comando.

En la construcción de cada una de estas matrices es posible la construcción de indicadores específicos de resultados (en relación con los objetivos específicos), de producto y cobertura  (para la acciones) y de ejecución presupuestaria.

El tablero de comando es la herramienta que permite el acceso a los reportes de situación a partir de la interrelación entre las matrices de datos que contiene y que condensan información sobre los problemas, los objetivos y las acciones.

La alimentación del sistema a través de un módulo de carga la realizan los equipos, los usuarios de carga; mientras que el uso mediante consultas está desarrollado en un módulo de consulta que permite observar el desempeño de los indicadores según la  información incorporada.

El Tablero de Comando expresa en su módulo de consulta un sistema de tendencias (medida en que los valores obtenidos crecen, decrecen o se mantienen sin cambio) y un sistema de semáforos que alertan sobre el alcance ("aceptable", "problemático", "deficiente" o "grave") de la meta propuesta.

El \ac{ssg} ha desarrollado también un módulo presupuestario capaz de implementar el seguimiento de un presupuesto construido mediante la metodología de presupuesto por programa.

Desde el año 2014, el equipo del \ac{ssg} se vinculó estrechamente con las \ac{ome}, una nueva herramienta de monitoreo, focalizado en los objetivos y metas de las jurisdicciones.

%\subsubsection*{Situación actual}

%La diseminación de la metodología  de trabajo con el SIG y la aplicación de sus herramientas ha avanzado en dirección a una aplicación acotada a los organismos y/o jurisdicciones que lo demandan, que ha limitado su  pretensión de implementación en toda la Administración Pública Nacional. 

%La aplicación del SIG requiere una fuerte tarea de sensibilización (a nivel de los decisores), capacitación y asistencia técnica a nivel de los ejecutores, de allí que los avances en su implementación remiten a los avances en las etapas de sensibilización (en mayor escala) y de capacitación y asistencia técnica en la preparación de distintas de la metodología.

%El efecto demostración a partir de la utilización del SIG en la Secretaría de Gabinete y en el Ministerio de Trabajo, ha permitido atraer el interés de otros organismos tales como el INDEC, la Subsecretaría de Asuntos Técnico Militares del Ministerio de Defensa, el Programa de Fortalecimiento Institucional del Honorable Senado de la Nación, y la Subsecretaría de Emergencias del Gobierno de la Ciudad Autónoma de Buenos Aires.

%El programa presenta como experiencias más significativas algunas de las siguientes:

%    \begin{itemize}
%        \item La experiencia realizada, con el Ministerio de Trabajo, Empleo y Seguridad Social iniciada en 2004 y que permitió la formulación del Plan Estratégico 2005 -2007 y la aplicación del SIG en su monitoreo. Para 2008 Se elaboró el Plan Estratégico 2008 – 2011, que permitió el desarrollo de ajustes y revisiones permanentes del plan, los indicadores, la metodología y las herramientas.
%        \item Secretaría de Gabinete y Gestión Pública de la Jefatura de Gabinete de Ministros. La misma permitió la elaboración de un Plan estratégico para una Gestión  Pública de Calidad (208 – 2010)
%        \item Subsecretaría de Emergencias del Gobierno de la Ciudad Autónoma de Buenos Aires. En 2006 y 2007 se trabajó en la aplicación de la metodología de Matriz de problemas, objetivos y acciones.
%    \end{itemize}

%Al momento de la entrevista con los Equipos del SISEG se comentó la idea de formulación de la modalidad de implementación del SIG, que permitiera pasar de la sensibilización mediante un módulo de demostración a una utilización más intensiva por las unidades que demandaran el sistema. Las condiciones para el acceso a la metodología y a la tecnología, si avanzara esta perspectiva, así como el uso de la información producida serían parte de nuevos acuerdos con las jurisdicciones.

\subsection{\acrlong{ome}}

A fines del año 2013, la gestión de la \ac{jgm} intentó profundizar el proceso de monitoreo de las actividades de la \ac{apn}, y creó lo que se denominó \acrfull{ome}. Con un fuerte impulso político, la \ac{jgm}, en conjunto con los diferentes ministerios de la \ac{apn} consolidó objetivos y metas planteados por cada uno de ellos para el año 2014. El proceso comenzó sin soporte informático, pero durante el año 2014 se desarrolló un sistema de de carga descentralizado con un módulo de explotación de datos que permite una clara y rápida visualización de la información.

El proceso de definición de metas tiene los siguientes objetivos:
    \begin{enumerate}
        \item Definir y consolidar objetivos estratégicos de política pública y metas de resultado
        \item Contribuir a la instalación de criterios comunes de planificación y programación en torno a resultados de gestión.
        \item Producir información sistemática y confiable sobre el estado de avance en el cumplimiento de \ac{ome} sustantivos de gestión.
    \end{enumerate}
    
Esta iniciativa, superadora del \ac{ssg} en el alcance jurisdiccional, busca crear un entorno de coordinación interministerial, para sistematizar de manera anual, los propósitos institucionales de las dependencias de las \ac{apn}. A partir de la definición de estas metas, se crea un dispositivo de información sobre la planificación, que permite informar periódicamente al Jefe de Gabinete y a las distintas autoridades, sobre el estado de situación de las acciones, de forma trimestral.

Las \ac{ome} incluyeron, en el año 2014, objetivos y metas de 13 ministerios y 4 organismos descentralizados (\acrshort{afip}, \acrshort{anses}, \acrshort{incaa}, \acrshort{sedronar}).

Esta nueva instancia de recolección de metas y objetivos se profundizó en el 2015, y se espera continúe en el 2016.%, teniendo año a año perfecciones metodológicas que ponen en valor el sistema creado por \ac{jgm}.

\subsection{\acrfull{sintys}}

El \ac{sintys} (Sistema de Identificación Nacional Tributario y Social) se crea en 1998 por el Decreto 812, en el ámbito de la Jefatura de Gabinete de Ministros en 1998 y financiado a través de un convenio de préstamo con el \ac{bm}, fue ratificado por la ley 25.345, conocida como ``ley antievasión''. Por su parte, el Pacto Fiscal, aprobado por la ley 25.400, contó con el compromiso de todos los gobernadores de asistir al \ac{sintys} y de crear dispositivos similares al \ac{sintys} en cada una de las provincias. 

El Programa ha evolucionado en dirección a la producción de información social y patrimonial de las personas, y cobra relevancia como sistema de provisión de información, a partir de procedimientos altamente calificados para coordinar  el intercambio de información al interior del Estado. %El \ac{sintys} se define como una red de interconexión de datos que tiene como objetivo comunicar a todas las bases de datos que existen en el país, tanto a nivel municipal, provincial como nacional. 

El \ac{sintys} tiene por misión contribuir a la mayor eficacia y eficiencia de la inversión social y al mejor cumplimiento tributario, coordinando el intercambio de información en función de sus atribuciones legales y de los acuerdos suscriptos con las provincias, los municipios y los organismos nacionales.

El organismo se define como una red de interconexión de datos que tiene como objetivo comunicar a todas las bases de datos que existen en el país, tanto a nivel municipal, provincial como nacional. 

Procura atender algunos de los siguientes problemas relativos a la información  con que cuenta el sector público, entre ellos: la desarticulación entre jurisdicciones, su fragmentación y escasez para la toma de decisiones, y la aplicación  de  criterios disímiles de tratamiento

En este sentido, el \ac{sintys}  se presenta como una herramienta de gestión eficiente que, para el  sector público:
    \begin{itemize}
        \item Facilita la identificación unívoca y homogénea de las personas y la adopción de estándares de intercambio de información gubernamental.
        \item Permite el acceso a los atributos sociales y fiscales mediante la coordinación de Información relacionada con jubilaciones y pensiones y de programas sociales, cobertura de salud, educación, empleo, información patrimonial e impositiva.
        \item Proporciona sólo aquella información que sea de incumbencia del organismo solicitante,
    \end{itemize}

Los usuarios del \ac{sintys} son todos aquellos organismos nacionales, provinciales y municipales que adhieren al Sistema mediante diferentes tipos de acuerdos. Todos ellos acceden a los siguientes servicios derivados del intercambio institucionalizado de información: 
    \begin{itemize}
        \item Validación de datos de personas. 
        \item Identificación del CUIT/CUIL/CDI de las personas. 
        \item Control de supervivencia. 
        \item Elegibilidad para beneficios sociales. 
        \item Detección de incompatibilidades y pluricobertura en la percepción de beneficios sociales, previsionales, de salud y vivienda. 
        \item Verificación de situación laboral (empleo público y privado). 
        \item Detección de posible incumplimiento tributario (impuestos nacionales, provinciales y municipales). 
        \item Comparación de los atributos de identidad de las personas registrados en las bases provinciales con los de padrones nacionales. 
    \end{itemize}

Además, el \ac{sintys} lleva adelante proyectos especiales orientados a la actualización e informatización de registros públicos, en temáticas estratégicas tales como: Registros de Personas Jurídicas Provinciales; Registros de la Propiedad Inmueble Provinciales; Registros Civiles Provinciales; y Generación de \ac{lua}.

En el año 2002, el programa es transferido de la \ac{jgm} al \ac{cncps}, dependiente de la Presidencia de la Nación, y por medio del \textcite{decreto78} se convierte en Dirección Nacional del Sistema de Identificación Nacional Tributario y Social



%\subsubsection*{Situación actual}

El \ac{sintys} presenta hoy, en el conjunto de sistemas de información, un desarrollo ponderado y basado en avanzados criterios de protección de los datos y de tecnología apropiada al cumplimiento de su misión.

Los principales usuarios del sistema son los programas sociales. La valiosa información social y patrimonial de las personas que puede cruzar el sistema se convierte en una herramienta moderna para el seguimiento de la asignación de beneficios o la percepción de derechos, de allí que su uso sea extendido y confiable en cuanto a la información que permite a los programas la validación de sus asignaciones.

Al contener información de grandes bases de datos como la de \ac{afip}, \ac{anses}, \ac{renaper}, registros de propiedad inmueble provinciales, etc., el \ac{sintys} presta al Estado un servicio importante en términos de facilitar el acceso a datos de las personas, particularmente información social y patrimonial. 

\subsection{\acrfull{siempro}}

Creado en el año 1995 por \citeauthor{resolucion2851} de la ex Secretaría de Desarrollo Social,  surge con las características de las herramientas de producción de información que se generaban a partir de las necesidades de los programas financiados por los organismos internacionales de crédito; entre ellas la producción de información para el seguimiento de los programas financiados por estos organismos. En sus modalidades y con aquella impronta, expresa desde sus inicios una fuerte presencia institucional y por lo tanto resulta relevante por su contribución a los diseños de instancias de monitoreo y evaluación de los programas y proyectos focalizadas o de combate a la pobreza. 

En el contexto del surgimiento de estas iniciativas, se encuentran consideraciones relativas a la preocupación por el desarrollo de políticas sociales focalizadas, para lo cual era imprescindible contar con información acerca de las condiciones de vida de la población y, en particular, de la población más carenciada. Junto con el \ac{indec} se desarrolló un Sistema de Información Social para la, entonces, Secretaría de Desarrollo Social y en 1998, comenzó a instrumentarse el \ac{sisfam}, para la construcción de un padrón único de beneficiarios de programas sociales y llevar a cabo un censo de beneficiarios potenciales de programas sociales.

El \ac{siempro} tiene como objetivos: 

    \begin{itemize}
        \item Establecer un sistema de información, evaluación y monitoreo de los programas sociales nacionales.
        \item Desarrollar e implementar el Sistema de Identificación y Selección de Familias Beneficiarias de Programas y Servicios Sociales (SISFAM).
        \item Fortalecer a las áreas sociales nacionales y provinciales en el desarrollo e instalación de sistemas de monitoreo y en la realización de evaluaciones.
        \item Producir nueva información a través de la Encuesta de Desarrollo Social, Condiciones de Vida y Acceso a Programas y Servicios Sociales.
        \item Asegurar la disponibilidad de la información necesaria sobre la población en situación de pobreza y vulnerabilidad social y la ejecución de los programas sociales dirigidos a atenderla.
        \item Capacitar a funcionarios y técnicos de las áreas nacionales y provinciales en política y gerencia social.
        \item Diseminar y transferir metodologías y sistemas de información a las agencias nacionales y provinciales.
        \item Fomentar la vinculación entre el sector académico y el Estado en la investigación y desarrollo de políticas y programas sociales.
    \end{itemize}

Por otro lado, a partir de 1998 comenzó a instrumentarse \ac{sisfam}, siendo coordinado en conjunto con la estrucura del \ac{siempro}. Las líneas de acción del \ac{sisfam} son:
    \begin{itemize}
        \item Construcción del Padrón Único de Beneficiarios de los Programas Sociales.
        \item Censo de Beneficiarios Potenciales de Programas y Servicios Sociales.
    \end{itemize}
% Actualmente, las actividades del Área de Información Social, se centran en la construcción, sistematización y análisis de  indicadores específicos de demanda social. Es, además,  productor de información a partir de datos propios del \ac{siempro}, generados en herramientas como las encuestas y relevamientos realizados por el SISFAM, el seguimiento de indicadores a partir de la utilización de información estadística del SEN, y de los operativos propios para la recolección de información que demandan las políticas de inclusión. 

En términos de estructura, el \ac{siempro}, partir de diciembre de 1999,  pasa a depender de la Secretaría de Tercera Edad y Acción Social, la cual pertenece al Ministerio de Desarrollo Social y Medio Ambiente. Para ese momento, el Gabinete Social funciona en el ámbito de la \ac{jgm}.

Finalmente, en 2002 y mediante el \citeauthor{decreto357} pasa a depender del \ac{cncps} y a partir de 2007 mediante el \citeauthor{decreto78}, obtuvo el rango de Dirección Nacional de Sistemas de Información, Monitoreo y Evaluación de Programas Sociales, pasando el \ac{sisfam} a formar parte de esta Dirección Nacional. 

Actualmente está conformado por 4 áreas: \ac{sisfam}, Monitoreo de Programas Sociales, Evaluación de Programas Sociales, Análisis e Información Social e Informática.

%\subsubsection*{Situación actual}

Desde 2006, el \ac{siempro} ha replanteado sus actividades en función de la nueva realidad del país y de las políticas sociales. El diseño institucional del \ac{siempro} en su calidad de programa con financiamiento externo, contemplaba una estructura de áreas de trabajo conducidas por la figura de gerentes (monitoreo y evaluación, información social e informática) y un cuerpo de consultores especialistas en cada una de esas áreas. Se ha privilegiado, a partir de 2006,  la consolidación de los equipos propios en la temática de la evaluación, diferenciando esta propuesta de las orientaciones que traía el organismo desde sus orígenes, marcados por el suministro de estos servicios mediante contratación externa.

Los bienes y servicios que ofrece son:
    \begin{itemize}
        \item Georeferenciamiento de información social.
        \item Relevamiento de la población beneficiaria en situación de pobreza y vulnerabilidad social.
        \item Evaluación y monitoreo de programas sociales.
        \item Encuesta sobre condiciones de vida.
        \item Informes de situación social y pobreza.
        \item Bases de datos de indicadores sociales.
        \item Base de datos de programas sociales, nacionales y provinciales.
        \item Registros de beneficiarios.
        \item Sistema de identificación y selección de familias beneficiarias actuales y potenciales de programas sociales.
        \item Asistencia técnica y capacitación.
    \end{itemize}
    
Como insumo de los productos ofrecidos, el \ac{siempro} ha desarrollado herramientas y sistemas para el Monitoreo y la Evaluación. 

Respecto al Monitoreo, se ha desarrollado una herramienta de monitoreo denominada \ac{sim}. En sus comienzos era un software descentralizado en cada una de las jurisdicciones que requería la agregación manual de datos. Posteriormente, y a partir del 2008, ha evolucionado a un entorno centralizado, accesible desde un navegador web estándar que permite a los programas la carga y consulta \emph{on line} de la información. Las formas de articulación institucional con los programas y de integración de la información de monitoreo, se guía por las prioridades que establece el \ac{cncps}. Así, el sistema recoge información sobre la ejecución del programa para un periodo determinado y con distintos niveles de localización de la ejecución del programa (municipios, departamentos y provincias). La información sobre metas del programa permite constatar el desempeño del mismo en un periodo determinado de ejecución. Entre los productos originados en el \ac{sim}, el \ac{siempro} brinda a las autoridades de los ministerios y organismos que integran el \ac{cncps}, reportes con información provincial o local, con el respectivo geo-referenciamiento, de los montos ejecutados por programas, el total de prestaciones realizadas en un determinado período y la cobertura de las mismas expresada en cantidad de beneficiarios. En el Apéndice \ref{apendice-siempro} se presenta en detalle los programas que, hasta el año 2013, reportaban al \ac{sim}
En relación a la Evaluación, desde sus orígenes, el \ac{siempro} ha construido una valiosa experiencia en el campo de la evaluación de programas y proyectos, en la que están contenido un conjunto de evaluaciones de programas, en  gran parte con financiamiento internacional, realizadas mediante equipos externos contratados a tal efecto. Muchas de estas evaluaciones, realizadas por Universidades Nacionales, se listan en el Apéndice \ref{apendice-siempro}. 

En la actualidad, el \ac{sim} recibe información de los siguientes ministerios que ejecutan políticas sociales: Ministerio de Desarrollo Social, Ministerio de Planificación Federal, Ministerio de Trabajo, Ministerio de Agricultura. Alrededor de 50 programas sociales nacionales, proveen de información sobre las prestaciones, la cobertura, y los recursos movilizados (detallados en el Apéndice \ref{apendice-siempro}). Por otro lado, esta información no se encuentra disponible en su página web. Los documentos resultantes de las evaluaciones no están disponibles. La información social elaborada por el organismo puede ser solicitada por otros organismos públicos así como también por las instituciones que integran el \ac{cncps}, entre ellos su Consejo Asesor integrado por Organizaciones No Gubernamentales.

\subsection{\acrlong{pepp}}

Con el objetivo de institucionalizar los procesos de evaluación de políticas públicas en la Administración Pública Nacional y potenciar las capacidades para su desarrollo, en el año 2013 la Jefatura de Gabinete de Ministros crea (por Resolución Nº 416/2013)  el \acrfull{pepp}.
Este programa se propone promover la sensibilización, la consolidación en agenda e institucionalizar la evaluación de las políticas públicas en la administración pública nacional; fomentar la investigación aplicada, comparada y participativa. También pretende diseñar metodologías y herramientas de evaluación de políticas públicas susceptibles de ser aplicadas en los organismos gubernamentales; desarrollar capacidades para el diseño e implementación de diversos tipos de evaluación de programas y proyectos; evaluar programas, proyectos y/o políticas implementadas en el ámbito de la administración pública nacional de manera conjunta y en coordinación con los organismos que ejecutan dichas intervenciones, y/o asistir técnicamente para su desarrollo.
Para el cumplimiento de sus objetivos desarrolla acciones en torno a cuatro ejes centrales:

    \begin{itemize}
        \item desarrolla procesos concretos de evaluación de políticas públicas nacionales; 
        \item desarrolla capacidades en materia de evaluación en la Administración Pública Nacional; 
        \item establece directrices y generar conocimientos mediante investigación aplicada en evaluación de políticas públicas y; 
        \item promueve la sensibilización, la consolidación en agenda e institucionalización de la evaluación de políticas públicas en la Administración Pública Nacional.
    \end{itemize}

El programa se encuentra ubicado bajo la órbita de la Secretaría de Gabinete y Coordinación Administrativa pero, debido a su experiencia en procesos vinculados al monitoreo y la evaluación de programas, proyectos y políticas públicas, la conducción operativa está en manos de una Unidad Ejecutora integrada por las Subsecretarías de Gestión y Empleo Público, la Subsecretaría de Evaluación del Presupuesto Nacional y la Subsecretaría de Evaluación de Proyectos con Financiamiento Externo.



%La pertienencia del programa en la estructura de \ac{jgm} tiene sustento en Ley de Ministerios (Ley 22.520), que le otorga entre otras atribuciones, capacidad para:
%    \begin{itemize}
%        \item Coordinar y controlar las actividades de los Ministerios y de las distintas áreas a su cargo, realizando su programación y control estratégico, a fin de obtener coherencia en el accionar de la administración e incrementar su eficacia.
%        \item Entender en la difusión de la actividad del Poder Ejecutivo Nacional, como así también la difusión de los actos del Estado Nacional a fin de proyectar la imagen del país en el ámbito interno y externo.
%    \end{itemize}

Desde 2013 el Programa ha dictado capacitaciones para el personal de la \ac{apn}, ha realizado el "Seminario Internacional de Evaluación", con la presencia de expertos internacionales,  publicó en el sitio web de \ac{jgm} el Banco de evaluaciones de Políticas Públicas, financió de 6 Evaluaciones en diversos organismos y se ha redactado el “Manual de base para la Evaluación de Políticas Públicas”, consensuado con las mas importantes áreas de evaluación de diferentes organismos.


