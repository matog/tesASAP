\newpage
\section{Articulación los sistemas de información, evaluación y monitoreo} \label{articulacion}

A lo largo de la Sección \ref{sistemas} se ha detallado el funcionamiento de los principales sistemas de información, monitoreo y evaluación generados a partir de la Reforma del Estado. Estos organismos, a pesar de no formar parte de un sistema formalmente creado, son partes fundamentales en la instauración de la \ac{gpr} en la \ac{apn}

Por este motivo, la \ac{apn} cuenta con valiosas herramientas para desarrollar y implementar un política de \ac{gpr} basadas en las fortalezas de estos sistemas. La fuerte implantación institucional y los desarrollos tecnológicos existentes son dos ventajas con las que cuenta la \ac{apn} a la hora de analizar los pasos a seguir. Por otro lado, es conveniente potencias algunos aspectos para que la \ac{gpr} se transforme un ciclo armonioso, fundamentalmente aquellos referidos a la vinculación del plan-presupeusto.

\subsection{Fortalezas de los sistemas de información, evaluación y monitoreo}
    \begin{itemize}
        \item La implantación institucional de los sistemas: La ubicación de los sistemas de información, evaluación y monitoreo en áreas estratégicas para la toma de decisiones de política pública, resulta un aspectos destacado y por lo tanto una fortaleza de las experiencias y desarrollos institucionales en este campo. En el campo de las políticas sociales, la concentración de gran parte de la función de monitoreo y evaluación en el \ac{cncps} inviste formalmente a estos instrumentos de una funcionalidad relevante, dado que es el ámbito en el que se diseñan las políticas sociales. En relación con la capacidad de monitoreo y evaluación de la gestión, la inserción de esta función en el ámbito de la Secretaría de Gabinete, en tanto colaboradora directa del Jefe de Gabinete, se visualiza como coherente y formalmente potente para proveer de información sobre la marcha de las políticas planes y proyectos. La situación del \ac{pepp} es similar, ya que su pertenencia a la \ac{jgm} le otorga la capacidad de articulación entre los actores de la temática. La capacidad de coordinación e integración de los sistemas de información, evaluación y monitoreo, estaría en condiciones de ser profundizada. Los estilos de gestión son, en todo caso, los proveedores de estímulos o frenos a estas potencialidades. Es decir, el valor que la gestión dé a la producción de información sistemática y periódica sobre la marcha de sus acciones servirá de orientación para profundizar en el desarrollo de los sistemas.
        \item Los desarrollos tecnológicos y conceptuales: Siendo que los avances en el campo de las tecnologías de información y transmisión de datos son cada vez más accesibles y forman parte de la exigencias de vinculación al medio, los recursos humanos de los organismos pasan a ser un componente prioritario.  Las capacidades profesionales y técnicas de los organismos involucrados (\ac{siempro}, \ac{sintys}, \ac{ssg}, etc.) facilitan la introducción de innovaciones y perspectivas relativas a la mejora de la gestión de estos sistemas. Hay una notable experiencia acumulada en la producción de información para el monitoreo y la evaluación en el propio sector público.
    \end{itemize}

\subsection{Aspectos a mejorar en los sistemas de información, evaluación y monitoreo}

    \begin{itemize}
        \item Integración transversal de los sistemas. La integración transversal de los sistemas de información, evaluación y monitoreo con los respectivos procesos planificación, presupuesto y control son incipientes y aún insuficientes.
        \item En relación con el presupuesto. En todos los casos, es notoria la falta de articulación con el presupuesto nacional y con las áreas que producen esta información.  Los déficit y dificultades remiten a aspectos de integración y coordinación que dificultan el acceso a la información actualizada de la ejecución presupuestaria; otras, a dificultades técnicas relativas a la necesidad de compatibilización de formatos, dimensiones y elementos propios del tratamiento y procesamiento de la información presupuestaria y su homologación con los sistemas de monitoreo y evaluación. 
        \item En relación con el planeamiento. La articulación a este nivel en el mejor de los casos se da a nivel de programas o proyectos. En el caso del SIG, el sistema se origina idealmente en el plan estratégico de un organismo o un plan o un proyecto. En el caso del SIEMPRO, el sistema de monitoreo permite desagregar aspectos (prestaciones, unidades de medida, beneficiarios, presupuestos) y tomar las metas presupuestarias para verificar los avances, si bien esta actividad  no está directamente relacionado con el momento de planificación de las políticas, programas  y/o proyectos.
    \end{itemize}

\subsection{Funcionamiento de los sistemas}
    \begin{itemize}
        \item La provisión de información: la alimentación de los sistemas tienen las dificultades propias de una articulación débil entre programas aún de un mismo ministerio. En el caso del \ac{siempro}, una buena cantidad de programas y proyectos proveen de información de la ejecución, si bien en muchos casos, dicha provisión no es regular. En el caso del \ac{ssg}, actualmente, es pequeño el número de programas el que utiliza el sistema. La información producida, en relación con el desempeño del programa que lo está aplicando, es importante para el propio organismo productor de la información. Al ser acotado el universo de programas y jurisdicciones que lo utilizan, su contribución al seguimiento estratégico como el que se propone en sus objetivos, es escasa aún. 
        \item Los usuarios: En el caso del \ac{siempro}, es el propio \ac{cncps} el que utiliza la información, los ministerios, organismos, instituciones que lo integran son sus principales usuarios.  Habitualmente los ministerios, en particular el Ministerio de Desarrollo Social, solicitan reportes de ejecución de planes y/o proyectos en algunas áreas del país. El organismo puede responder ágilmente a dichas consultas. Otros usos, en particular los accesos públicos a información no están disponibles. En el caso del \ac{ssg}  el principal usuario es el propio programa productor de a información, o sus áreas jerárquicas de dependencia. En el caso del \ac{pepp}, los usuarios son programas u organismos que voluntariamente se ofrecen a ser evaluados, pero con resultados optimistas para el futuro dado su reciente creación.
    \end{itemize}


