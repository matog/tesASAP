\newpage

\section*{Prólogo}
\addcontentsline{toc}{section}{Prólogo}
La presente tesina se realizó en el marco de la Especialización en Gestión Pública por Resultados dictada por la Facultad de Ciencias Económicas de la Universidad de Buenos Aires y la Asociación de Presupuesto y Administración Financiera Pública.

Tanto la \ac{gpr} como los sistemas de información, monitoreo y evaluación han tomado gran importancia en la \ac{apn}, que busca modernizar sus sistemas de gestión. Por tal motivo, el tema elegido cobra una gran importancia a la hora de analizar las herramientas utilizadas por el Sector Público.

A efectos de lograr un acabado análisis de las herramientas generadas a partir de la impronta de la \ac{gpr}, se toma como punto de partida aquellos organismos, programas o proyectos creados a partir de la denominada Reforma del Estado, en la década del '90, y profundizada en los últimos 10 años.

Debido a mi trayectoria profesional en varios de los organismos que impulsan estás nuevas actividades, la tesina abunda en información obtenida en entrevistas realizadas de manera personal, un detallado análisis normativo y en una seleccionada bibliografía teórica.

Por último, un especial agradecimiento al tutor de esta tesina, el Lic. Norberto Perotti, quien apoyó en todo momento y firmemente la elaboración del presente documento.