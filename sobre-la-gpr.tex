\newpage
\section{Sobre la Gestión por Resultados} \label{gpr}
\subsection{Marco conceptual}

La \ac{gpr} forma parte de la agenda pública desde que se produjeron importantes transformaciones en la relación entre el Estado y la sociedad en las últimas décadas. Sin embargo, no existe ni un modelo ni un modo de implementación unívoco de los distintos instrumentos y es, precisamente, en estas diferencias donde se expresan el tipo de Estado que se busca y el compromiso que el Estado asume con los ciudadanos.

El inicio de este proceso debe rastrearse en la década de 1980, cuando se pusieron en marcha reformas administrativas-gerencialistas en los sectores públicos de países como Nueva Zelanda, Reino Unido y Australia y cuyo eje fue la modificación de la principal herramienta que tiene el Estado para asignar recursos, el sistema de presupuesto público, con el objeto de pasar de un sistema orientado a los gastos a uno orientado a la búsqueda de resultados \parencite{arellano1999}.

El paradigma de la \ac{ngp} promovió la modernización de las administraciones públicas. En América Latina, a posteriori de las reformas orientadas a reducir el tamaño del Estado y sus funciones, comenzó la preocupación por reformar el Estado ``hacia adentro'' \parencite{oszlako1999}. Estos procesos se sustentaron también en el diagnóstico realizado por los organismos multilaterales de crédito que a partir del año 1997 comenzaron a promover una serie de medidas destinadas a mejorar la administración pública, fundándose en la necesidad de incrementar las capacidades institucionales y de gestión estatal como premisa para lograr un buen funcionamiento de los mercados.

Como todo paradigma, la \ac{ngp} presenta matices en su interior. En su versión más ortodoxa plantea transpolar técnicas de management o gerenciamiento, del sector privado al sector público, presentadas como neutrales políticamente y aplicables a cualquier contexto y organización, con la finalidad de reorientar el servicio público hacia la demanda, bajo criterios de eficiencia, eficacia y economía.

Con todo, la necesidad de planificar, obtener resultados y sustentar el mecanismo de toma de decisiones en un diagnóstico científicamente fundado de la situación existente, no es ajena a la tradición de la estatalidad.
Según \citeauthor{garcialopez2010} (\citeyear{garcialopez2010}), la \ac{gpr} es una estrategia de gestión pública que conlleva tomar decisiones sobre la base de información confiable acerca de los efectos que la acción gubernamental tiene en la sociedad. Varios países desarrollados la han adoptado para mejorar la eficiencia y la eficacia de las políticas públicas. En América Latina y el Caribe, señalan estos autores, los gobernantes y administradores públicos muestran un interés creciente en esta herramienta de gestión. En tal sentido, la \ac{gpr} es una estrategia integral que toma en cuenta los distintos elementos del ciclo de gestión (planificación, presupuesto, gestión financiera, gestión de proyectos, monitoreo y evaluación), subrayándose el papel que desempeñan estos elementos en la creación de valor público.

La consolidación de este nuevo modelo de gestión pública, requiere una modificación de la cultura administrativa que admita una definición clara de los objetivos de la organización, focalizados en sus aspectos sustanciales y no como procesos administrativos formales, de manera que la evaluación de la gestión pública se realice a través del cumplimiento de metas más que a partir, solamente, del respeto a reglas que, en muchas ocasiones, son auto-referidas por la burocracia. Se espera que de esa manera se fortalezca la capacidad de la administración pública para aprender de su desempeño y mejorar continuamente la prestación de servicios públicos. 

Siguiendo a la clasificación elaborada por el \ac{sep} \parencite{sepprodev}, la \ac{gpr} divide al ciclo de gestión en cinco pilares indispensables para que el proceso de creación de valor público esté orientado a lograr resultados:

    \begin{enumerate}
        \item planificación para resultados, 
        \item \ac{ppr}, 
        \item gestión financiera, auditoría y adquisiciones, 
        \item gestión de programas y proyectos y 
        \item monitoreo y evaluación. 
    \end{enumerate}

\subsection{La Gestión por Resultados en Argentina}

La implementación de la \ac{gpr} en Argentina se remonta a fines de la década del '80 y principios de la década del '90. En esa época tuvieron lugar algunas experiencias en torno al planeamiento estratégico y a la reorganización de estructuras de organismos públicos, que en general no fueron formuladas en forma coordinada, generalizada ni sistemática. Principalmente estuvieron marcadas por esfuerzos atomizados de algunos organismos, que no fueron acompañados por procesos impulsados por la gestión política del país.

La formalización de todos estos procesos vio la luz recién en el año 2001, cuando por medio del \citeauthor{decreto103} (\citeyear{decreto103}) se creó el Plan para la Modernización del Estado, que impulsó de manera formal la \ac{gpr} en la \ac{apn}.
Como se describe en la Sección \ref{marco-normativo}, este decreto establecía la elaboración de indicadores de gestión, Acuerdos Programa y sistemas de incentivos por desempeño institucional.

Este proceso, iniciado al menos formalmente, se vio interrumpido por la grave crisis que atravesó el país durante ese mismo año y los avances en la materia fueron escasos, a pesar que la legislación continúe vigente a la fecha.

Mas allá de algunas acciones aisladas, la \ac{gpr} como sistema recién volvió a tomar relevancia en el año 2009,  con la \textcite{cartaprodev} firmada entre el \ac{bid} y el \ac{mecon} (ratificada por \textcite{decreto1673}), que formalizaba la cooperación técnica no reembolsable a para la Cuenta A del \ac{prodev}.


