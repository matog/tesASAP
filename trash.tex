%\emph{\ac{siempro}:} Creado en el año 1995 por Resolución 2851/95 \footnote{CITAR} de la ex Secretaría de Desarrollo Social,  surge con las características de las herramientas de producción de información que se generaban a partir de las necesidades de los programas financiados por los organismos internacionales de crédito; entre ellas la producción de información para el seguimiento de los programas financiados por estos organismos. En sus modalidades y con aquella impronta, expresa desde sus inicios una fuerte presencia institucional y por lo tanto resulta relevante por su contribución a los diseños de instancias de monitoreo y evaluación de los programas y proyectos focalizadas o de combate a la pobreza. 

%Esta iniciativa, atada a las orientaciones del financiamiento internacional, se modifica a partir de una nueva visión política que se desarrolla a partir del gobierno que se inicia en el año 2003, en la que se adopta una concepción de políticas universales que garantizan derechos. El Estado se instala en el centro del entramado social con una perspectiva de construcción de ciudadanía. En esta visión los instrumentos de gestión pierden una pretendida autonomía y neutralidad ideológica, para ponerse al servicio de los grandes lineamientos de la política nacional y sus modalidades de implementación.

%Esta perspectiva se consolida en 2006, al convertir al \ac{siempro}en una Dirección Nacional de Información, Evaluación y Monitoreo rescatando los efectos que tuvo en la instalación de la temática y los instrumentos de monitoreo y evaluación, y en el diseño institucional de la unidades provinciales \ac{siempro}, con presencia en 19 jurisdicciones; a excepción de La Pampa, Neuquén, San Luís, Catamarca y Santiago del Estero, que no cuentan con agencia, para ponerlo en línea con las nuevas necesidades da las políticas de inclusión social.

%\emph{\ac{sintys}:} Desde su creación por Decreto 812., en 1998, ha evolucionado en dirección a la producción de información social y patrimonial de las personas, y cobra relevancia como sistema de provisión de información, a partir de procedimientos altamente calificados para coordinar  el intercambio de información al interior del Estado. El \ac{sintys} se define como una red de interconexión de datos que tiene como objetivo comunicar a todas las bases de datos que existen en el país, tanto a nivel municipal, provincial como nacional. 

%\emph{\ac{ssg}:} En 2008, en el marco de la política de modernización del Estado, se construye el SIG-SISEG (Sistema Integral de Seguimiento y Evaluación de la Gestión), del cual se espera que contribuya a “definir, sistematizar, medir y redefinir indicadores de planes, programas y proyectos de las principales políticas que ejecutan las áreas sustantivas de gobierno, y utilizar esa información clave como insumo estratégico de la política implementada”

%\emph{\ac{sig}:} El SIG  de Políticas Sociales, es una construcción que tiene su origen en el  sistema de información geográfico que se va configurando a partir del Programa Nacional Mapa Educativo, el cual se viene desarrollando desde el año 2004 en el Ministerio de Educación de la Nación  en forma conjunta con los Ministerios de Educación de las 24 jurisdicciones.

%\emph{Monitoreo de los Objetivos de Desarrollo del Milenio:} Con apoyo del \ac{pnud} se realizan las  actividades de difusión, seguimiento y monitoreo de esta iniciativa global. El Consejo Nacional de Coordinación de Políticas Sociales es el organismo responsable de llevar adelante las acciones de seguimiento y publicación de los avances.

%\emph{Sistema Estadístico Nacional:} Desde fines de los años 60, la Argentina cuenta con un sistema de estadísticas nacionales que implementa el \ac{indec}. Es sin dudas,  el sistema de información más abarcativo y riguroso con que cuenta el país. La información producida en esté ámbito es, por su rigor estadístico y la frecuencia de sus series, la base obligada en la construcción o comparación de todos los sistemas de información. Todos los sistemas tienen que ser rigurosos, tiene que ver con la metodología que implementan. 

%El advenimiento de la democracia, permitió a partir de 1983 la construcción de indicadores sociales de alta relevancia para la orientación de las políticas públicas y de la política social, en particular. Los indicadores utilizados en la medición de la pobreza, por ejemplo, han permitido la construcción de datos relevantes para la definición de políticas así  como también, el  desarrollo de otros sistemas de información que toman como base estos aportes sistemáticos y regulares de producción de información. Así, podemos ver como  se incorpora en 1984, el método NBI (Necesidades Básicas Insatisfechas) como forma de medición, a partir de la elaboración del indicador compuesto de NBI en base al Censo Nacional de Población y Vivienda de 1980 (INDEC, 1984 \footnote{AGRAGAR A BIBTEX}). Mientras que,  el otro sistema de medición, la línea de pobreza del Gran Buenos Aires que se emplea en las estimaciones regulares del \ac{indec} surgió de un estudio realizado entre 1988 y 1990 en base a los resultados de la Encuesta de Ingresos y Gastos de los Hogares de 1985/86. 

%En Junio de 2011, se ha presentado el Plan Estratégico 2011 2015 de producción de estadísticas que “incluye tanto la puesta en marcha de nuevas encuestas, como así también innovaciones en programas tradicionales en función de captar las características actuales del contexto económico y social. Entre otros esta conformado por: Encuesta sobre Tecnología de la Información y la Comunicación (TIC), Encuesta Nacional Económica, Programa Nacional de Cartografía Unificado - SIG, Encuesta Permanente de Hogares, Marco de Muestreo Nacional de Viviendas, Cuentas Nacionales, Encuesta Nacional sobre la Prevalencia de Sustancias Psicoactivas 2011, Encuesta sobre Tabaquismo, Encuesta Nacional de Condiciones de Vida 2012, Encuesta de Condiciones de Vida de la Tercera Edad, Encuesta Nacional de Gastos de los Hogares 2012, Objetivos de Desarrollo del Milenio (ODM) y Encuestas Complementarias del Censo 2010”.

%El carácter multi-jurisdiccional del SEN queda establecido en el  Dec. Nro. 1831/93, en el que se detallan las Áreas de responsabilidad estadística de los Ministerios, Secretarias de Estado y Organismos Estatales.










%\subsection{Sistema Nacional de Estadísticas (SEN)}

%La presencia del INDEC y del SEN (Sistema Estadístico Nacional), acentúan la importancia de la estadística oficial para la decisión de políticas de desarrollo en el área económica, demográfica, social y ambiental 

%Desde su creación en 1968,  mediante Ley  Nro. 17622,  el INDEC dependió alternativamente de diversas instancias del Poder Ejecutivo Nacional, preservando  la independencia en la producción de las estadísticas oficiales. Hoy, depende de la Secretaría de Programación Económica y Regional  (Ministerio de Economía y Finanzas Públicas de la Nación).

%Bajo los principios de centralización normativa y descentralización ejecutiva el INDEC coordina el funcionamiento del SEN (Sistema Estadístico Nacional), confeccionando el Programa Anual de Estadística y Censos, y desarrollando metodologías y normas que aseguren la comparación de información procedente de distintas fuentes.

%El Sistema Estadístico Nacional  está integrado por los servicios estadísticos de los organismos nacionales, provinciales y municipales (23 provincias y CABA). Para el fortalecimiento del Sistema Estadístico Nacional (SEN), el INDEC trabaja en el desarrollo de estadísticas provinciales, regionales y municipales, asegurando su comparabilidad y contar con un espacio donde ofrecer las publicaciones del INDEC y captar las demandas y necesidades de los nuevos usuarios.

%La información estadística se obtiene a través de distintos métodos de captación: Censos nacionales, encuestas por muestreo y estadísticas de registro (estas últimas a partir de información procedente de registros administrativos como el Sistema Integrado de Jubilaciones y Pensiones, la Administración Nacional de Aduanas,  Registro Civil, Migraciones y Ministerio de Salud, entre otros).

%Los Censos nacionales: comprenden los del tipo demográfico (de Población, Hogares y Viviendas) o los referidos a recursos nacionales (Agropecuario, Económico, etc.)

%Entre las Encuestas es importante destacar: la Encuesta Nacional de Gastos de los Hogares (ENGHO); Encuesta Nacional Económica (ENE); Encuesta Nacional a Grandes Empresas (ENGE); Encuesta Permanente de Hogares (EPH); Encuesta Industrial Mensual (EIM); Encuesta de Turismo Internacional (ETI); Encuesta de Ocupación Hotelera (EOH); Encuesta Nacional de Factores de Riesgo (ENFR)

%Esta producción de información permite la construcción de indicadores e índices apropiados para la medición en distintas temáticas de carácter estratégico para el desarrollo económico y social.

%Entre los índices podemos mencionar algunos de los más importantes, tales como: el Índice de Salarios; Estimador Mensual Industrial (EMI); Estimador Mensual de Actividad Económica (EMAE); Indicador Sintético de Servicios Públicos; Índice de Precios al Consumidor (IPC); Evolución de la Distribución Funcional del Ingreso; Índice del Costo de la Construcción, Gran Buenos Aires, (ICC-GBA); Tasas de actividad, empleo y desempleo, entre otros

%Mientras que para el  Sistemas de indicadores: pueden destacarse: Sistema de Índices de Precios Mayoristas; Sistema de Cuentas Satélites; Derechos del niño y los adolescentes (SIISENA); Balanza de Pagos; Sistema de Indicadores Sociodemográficos, entre otros

%O los grandes clasificadores como: Clasificador Nacional de Actividades Económicas (CLANAE); Clasificador de Actividades Económicas para Encuestas Sociodemográficas (CAES 1.0); Clasificador Nacional de Ocupaciones (CNO), entre otros. 

\subsection{Sistema de Seguimiento de Metas Físicas}


Una instancia fundamental en la implementación del pilar presupeustario de la \ac{gpr}, el \ac{ppr} implica trabajar en el desarrollo y consolidación de instancias de evaluación del desempeño de los programas presupuestarios. En ese marco, la definición de las metas físicas y el desarrollo de indicadores de resultados que permitan medir la eficiencia y/o la eficacia de los programas contenidos en el presupuesto nacional.
Para esto, la Subsecretaría de Presupuesto del \ac{mecon} ha expandido y generalizado el uso del Sistema Informático de Administración Financiera e-SIDIF. Dicha herramienta informática optimiza la comunicación entre programas, organismos y órganos rectores, contribuyendo decididamente al modelo de \ac{gpr} en el ámbito de la \ac{apn}. Actualmente se encuentran en desarrollo tanto un sistema transaccional, como un sistema gerencial, el cual está desarrollado con software de inteligencia de negocios que permitirá a los programas, organismos y autoridades políticas la evaluación de las metas públicas y la óptima explotación de la información; fundamentalmente a través del análisis conjunto de diversos datamarts  de Ejecución Financiera y Física del Presupuesto Nacional. 

\subsubsection*{El presupuesto por programas en la Argentina y el Sistema de Seguimiento de Metas Físicas}  


En la APN, la lógica de clasificación presupuestaría en términos concepto, finalidad y programa se desprenden de la Ley Nº 24.156 y de las reglamentaciones del Decreto Nº 866/92 y el Decreto Nº 1.815/92 y las Resoluciones Nº 891/92 y Nº 1.109/92 del Ministerio de Economía. En ese sentido, el PpP ya tiene varios años de aplicación formal en la APN, expresado en el proceso presupuestario y en los documentos de soporte las vinculaciones entre los fines y los medios, lo que permite definir y hacer el seguimiento de las variables físicas y financieras, así como sus relaciones. A su vez, las categorías principales PpP (programa, subprograma, actividad, proyecto, obra) están definidas como procesos de producción de bienes y servicios públicos, los cuales ya se presentan como indicadores físicos para las instancias de seguimiento y evaluación de su eficiencia. 

Bajo ese esquema normativo, la aplicación técnica del PpP ha permitido a la DEP elaborar un sistema de seguimiento que desde 1994 constituye un instrumento que simplifica las tareas de monitoreo y evaluación, ya que permite contar con una estructura de datos que sujeta a una actualización periódica y sistemática facilita la articulación y el flujo de información.

La ONP, a través de la DEP, lleva a cabo el seguimiento trimestral físico-financiero que sirve de insumo para la confección de la Cuenta de Inversión Anual, el cual se encuentra publicado y a disposición de la ciudadanía . Adicionalmente en la Cuenta de Inversión se analizan las acciones y productos de los respectivos programas presupuestarios desde una perspectiva anual y con mayor detalle de las labores realizadas por las unidades ejecutoras.

No obstante, a partir de los informes trimestrales de 2010, se incorporan al seguimiento financiero del gasto por finalidad y función indicadores de resultado de tipo producto y son acompañados por un breve análisis de desempeño. Como antecedente en el año 2007, se incorporan los primeros programas con seguimiento de metas físicas que paulatinamente se irían expandiendo al resto de la APN.

En el año 2010, se expone un cuadro general con el desempeño físico financiero de todos los programas presupuestarios con seguimiento físico. Se expone allí, en forma resumida y sistematizada, la ejecución físico-financiera de los programas de acuerdo al clasificador institucional del gasto, conteniendo la siguiente información: finalidad función de cada programa, ejecución presupuestaria acumulada al trimestre y su comparación con igual período del ejercicio anterior, programación anual y trimestral de las metas físicas, ejecución física al trimestre y en igual período del ejercicio anterior, porcentaje de ejecución física y desvíos con respecto a lo programado y, finalmente, un detalle sobre las causas de estos desvíos, de acuerdo a lo manifestado por los propios responsables.

Por su parte el Sistema de Seguimiento físico (metas físicas de producción y/o avance de proyectos de inversión) cubrió 265 programas en 2010 (66,3\% de la cartera programática). 

En ese mismo año, los programas bajo seguimiento representaron el 70,0\% del total de los créditos finales correspondiente a gastos corrientes y de capital de la Administración Nacional y el 83,7\% de los 302.764 cargos ocupados. 

Asimismo, recientemente se incorporó al Manual de Formulación Presupuestaria, el formulario “F8bis- formulario de información respaldatoria de las metas físicas, producciones en proceso y otros indicadores”, el cual permite respaldar la información cuantitativa de proyección presupuestaria plurianual de metas, producciones en proceso y otros indicadores de gestión en el proceso de formulación presupuestaria. Este formulario solicita a las jurisdicciones y sus respectivas unidades ejecutoras de programas, para cada medición física que se presente en el Proyecto de Ley de Presupuesto, la siguiente información cuali-cuantitativa: 

\subsubsection*{e-SIDIF}

De las exigencias normativas a partir de la Ley Nº 24.156/92 de Administración Financiera, se desprende la necesidad de crear herramientas que flexibilicen la gestión, agilicen las operaciones de transacción y registro y supongan nuevos elementos que faciliten la gestión diaria de los niveles medios de la APN. Con esas premisas, desde la SH se han ido desarrollando avances considerables en aquello que refiere a los soportes informáticos a los cuales se les reconocen un rol significativo en el despliegue de una estrategia de \ac{ppr}. La implementación de nuevos sistemas informáticos ha ido generando, de forma paulatina, las condiciones para una transformación de las formas de gestión de la administración pública. 

Este proceso de renovación, bajo el signo de una centralización normativa y una descentralización operativa, se inició con la implementación y consolidación de un sistema integrado denominado Sistema Integrado de Información Financiera Local Unificado (SLU) el cual brindó mayor funcionalidad y mayor estabilidad en su funcionamiento. Su desarrollo comenzó en 1998 y en el año 1999 se comienza la migración tecnológica de algunos SAF hacia esta actualización tecnológica. Con esta iniciativa se buscó generar y consolidar una base de datos para el Registro y Control Presupuestario y una inducción a buenas prácticas de administración financiera. 

Sin embargo, aún los procesos de registro y aprobación presupuestaria no se realizaban simultáneamente, lo que siguió significando demoras en la aprobación presupuestaria desde la SSP hacia los organismos ejecutores. En ese marco, la SH comienza a desarrollar una estrategia superadora que diera cuenta de la mayor demanda de funcionalidades que la APN requería. Así, hacia 2005 comenzó una nueva iniciativa de mejoramiento de los soportes informáticos que llevarían al desarrollo del e-SIDIF, el cual unifica los momentos de registro y transacción a partir de un soporte web y diseñando para su articulación funcional con el PpR.

Las principales ventajas del e-SIDIF responden a un salto cualitativo en los sistemas informáticos. Los avances más significativos se encuentran en el orden de la generación de una base de datos única – lo que evita la desarticulación entre los procesos registro y transacción presupuestaria. A su vez, esto significó el desarrollo de una herramienta que unificaba las instancias de gestión, registro y control, ahora en un marco orientado a la GpR.
En 2006 se implementó el primer modulo del e-SIDIF. Este es el e-FOP, módulo de formulación presupuestaría, lo cual no es otra cosa que el inicio natural del ciclo presupuestario. Siguiendo una estrategia de despliegue gradual pero de visibilidad temprana y bajo impacto hacia los usuarios, uno de los avances más importantes de la modernización del SIAF fue la incorporación de las reglas del negocio de la administración financiera dentro del sistema informático, convirtiéndolo en una instancia de gestión en si misma por sobre sus funciones de herramienta de registro contable. 

El proceso de consolidación del e-SIDIF ha implicado una maduración de los sistemas transaccionales lo que ha permitido concebir nuevos saltos en los soportes de información financiera y sus interrelaciones con los procesos de gestión en la administración pública. Con esos fundamentos, el e-SIDIF tiene las condiciones para el desarrollo de un sistema de tipo gerencial en torno al despliegue de una estrategia de PpR bajo la incorporación de software de negocios de moderna generación.

En consecuencia, el software ya presenta un conjunto de funciones que se enmarcan en el modelo conceptual de PpR lo que trae aparejado un cambio cualitativo en las formas de gestión. Entre los aportes significativos del e-SIDIF al PpR, unos refieren a la autonomía que se le brinda al usuario en términos de la posibilidad de configurar los circuitos de procesamiento. 

Desde el punto de vista de los organismos, e-SIDIF permite a los organismos elaborar la programación del gasto en forma calendarizada así como las solicitudes de cuota de compromiso y devengado. Con el módulo de escenarios de programación de pagos los usuarios pueden incorporar de forma automática las órdenes de pago según escenarios generados a partir de los datos seleccionados. Esto no es un dato menor, dado que permite la asignación de cuotas y su reasignación en torno a objetivos prioritarios. Por su parte, la ONP tiene competencias para asignar las cuotas según las prioridades de las autoridades superiores y distribuir las restantes disponibilidades entre los requerimientos que hayan efectuado los organismos, lo que implica una optimización de la eficiencia de la asignación presupuestaria.

Las formas de navegación al interior del e-SIDIF se encuentran adaptadas a los requerimientos de sus usuarios y permiten la generación de información gráfica de forma automatizada, brindando autonomía operativa. A su vez, el e-SIDIF brinda la posibilidad de confeccionar tableros de control y alertas según las particularidades de los organismos involucrados.

Con la implementación del e-SIDIF se han desarrollado no sólo lo sistemas de información financiera sino que también se está incorporado un sistema de planificación, seguimiento y evaluación presupuestaria. A su vez, el mismo soporte informático lleva a las autoridades responsables a trabajar en términos de PpR a medida que se sigue la propia lógica del software.

Entre estos negocios se encuentra la posibilidad de formular indicadores de desempeño por parte de las unidades ejecutoras los cuales pueden ser relevados de forma instantánea por parte de los OR, permitiendo un seguimiento en tiempo real de los programas de forma oportuna y confiable. Esto de por sí es un salto cualitativo de las formas de seguimiento antes reseñadas y facilita el monitoreo, no sólo por parte de los OR sino también de las autoridades responsables de los programas y las jurisdicciones y entidades. 
Los sistemas de medición para la fundamentación del cumplimiento de las metas físicas y la ocurrencia de sus gastos se da de forma calendarizada. En esta línea, el e-SIDIF contempla la carga del sistema de indicadores de desempeño, los informes de causas de desvío y conclusiones, lo que facilita el relevamiento de los informes por parte de las autoridades de los organismos y los OR.
