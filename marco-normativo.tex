\subsubsection{Marco Normativo}\label{marco-normativo}

El marco normativo necesario para la aplicación de la \ac{gpr} está conformado, por un lado, por las normas referidas a la Reforma del Estado, y por el otro, por las vinculadas al sistema presupuestario, financiero y de control. En esta sección se enumeran y describen brevemente los decretos y normativas de menor rango que formulan el Plan para la Modernización del Estado, estableciendo explícitamente como uno de sus ejes principales a la \ac{gpr}. Posteriormente se presentan las normas referidas a las estructuras y procesos necesarios para su implementación.

\paragraph{Normativa vinculada a la Reforma del Estado:} Dada la naturaleza de la \ac{gpr}, los procesos que abarca, y la articulación jurisdiccional que plantea, la sanción de una ley por sí misma no determina su exitosa implementación en la \ac{apn}. Por consiguiente, la normativa citada a continuación es un compendio que rescata la legislación que busca modernizar y articular los pilares que forman parte del sistema a través de la incorporación de nuevas metodologías y técnicas, para generar un todo coordinado donde el sistema en general tienda a transformar la gestión de la \ac{apn} en una orientada en resultados.

    \begin{enumerate}
        \item \citeauthor{ley24156} (\citeyear{ley24156}): La \citetitle{ley24156} fue uno de los principales instrumentos normativos a partir del cual se inició el profundo proceso de reforma estatal.
        \item \textcite{ley24629}: dio comienzo a la Segunda Reforma del Estado, estableciendo algunas normas para la ejecución del presupuesto de la \ac{apn} -que complementaban lo previsto en la \citeauthor{ley24156} -como así también nuevos lineamientos con el objetivo de mejorar el funcionamiento y la calidad de los servicios prestados por el Estado. En esta normativa se especificaban también los lineamientos básicos a considerar para la elaboración de los planes estratégico.
        \item \textcite{ley25512}: denominada Ley de solvencia fiscal y Calidad del Gasto Público, establecía las medidas a partir de las cuales se deberían ajustar los poderes del estado nacional para administrar los recursos públicos. Estas medidas se centraban en:
            \begin{enumerate}
                \item Formulación del presupuesto general de la \ac{apn},
                \item Eficiencia y calidad de la gestión pública,
                \item Programa de evaluación de calidad del gasto,
                \item Presupuesto plurianual,
                \item Información pública y de libre acceso,
                \item Creación del fondo anticíclico fiscal.
            \end{enumerate}
        Esta ley autoriza al Jefe de Gabinete de Ministros a partir del año 2000, a realizar Acuerdos Programas con las Unidades Ejecutoras de programas presupuestarios, para avanzar en el proceso de reforma del Estado, aumentar la eficiencia y lograr mejorar la calidad de gestión. Así es como aparece por primera vez en la legislación la figura de Acuerdo Programa.
        \item \textcite{decreto103}: Aprueba el Plan Nacional de Modernización del Estado. Esta normativa proponía una transformación centrada en la elaboración de indicadores de gestión, Acuerdos Programa y sistemas de incentivos por desempeño institucional. Este Decreto puede considerarse como el antecedente fundacional en cuanto a la introducción de la \ac{gpr} en el ámbito de la \ac{apn}. Establece un sistema de incentivos otorgado en función del cumplimiento de objetivos prefijados. Se otorga facultades para que los organismos realicen modificaciones en sus estructuras organizativas.
    \end{enumerate}	

\paragraph{Normativa vinculada a Estructuras y Procesos:}
Además de las leyes generales que dan el marco para la reforma del Estado en la Argentina y que proporcionan el encuadre y direccionamiento de las políticas que se pueden vincular a la \ac{gpr}, existen otra normas que dan lugar a la formación de estructuras y el desarrollo de procesos en dicha materia.

    \begin{enumerate}
        \item \textcite{decreto558}: Crea la Unidad de Reforma y Modernización del Estado, que constituyó uno de los primeros antecedentes institucionales relevantes en cuanto a la introducción del planeamiento estratégico como metodología para la modernización del aparato público. Se conformó en el ámbito de la \ac{jgm}. Su función esencial consistía en concluir el proceso de reforma del Estado y elaborar un programa de modernización para el mismo.
        \item \textcite{decreto928}: Establece iniciar una transformación profunda e incorporar el planeamiento estratégico como método en cinco organismos descentralizados considerados clave: \acs{dgi}, \acs{anssal}, \acs{anses}, \acs{inssjp} y \acs{ana}, lo cual implicaba el diseño de un plan estratégico para cada uno de ellos. Este decreto instituye la obligatoriedad del diseño de dichos planes para los organismos descentralizados y postula un nuevo rol del estado, teniendo como eje principal la orientación al ciudadano, la medición de resultados y la jerarquización y participación de los recursos humanos.
        \item \textcite{decreto229}. Crea del Programa Carta Compromiso con el Ciudadano. El Programa tiene como principal finalidad mejorar la relación de la \ac{apn} con los ciudadanos, especialmente a través de la calidad de los servicios que ella brinda. El punto de partida para su implementación lo constituye la decisión de los organismos de comenzar a concebir y desarrollar los servicios públicos, con la mirada de quienes los utilizan o reciben. La Carta Compromiso con el Ciudadano intenta enmarcar la relación entre los ciudadanos y los organismos públicos, por lo que podría considerarse uno de los elementos materiales esenciales de la Reforma del Estado.
        \item \textcite{decreto673}: Crea la Secretaría de Modernización del Estado, organismo responsable de articular e implementar las medidas establecidas a partir de la llamada Segunda Reforma del Estado.
        \item Decreto 992/01: establece la creación de las Unidades Ejecutoras de Programas, las que tendrán a su cargo –bajo la responsabilidad de los Gerentes de Programa- la ejecución de Programas específicos en relación directa con actividades vinculadas a la obtención de resultados hacia la población objetivo. El mismo específica en su artículo 6º ``El Compromiso de Resultados de Gestión deberá especificar los resultados a alcanzar por el responsable de la Unidad Ejecutora de Programa, los recursos que se pondrán a disposición y los indicadores objetivos de desempeño, de manera convergente con las metas físicas establecidas para el programa en el Presupuesto Nacional, en el caso que las hubiera. Dichos indicadores integrarán el Sistema de Evaluación de Resultados, en el que también se especificará la periodicidad de las evaluaciones parciales de la labor de los Gerentes de Programa. Cuando la naturaleza de la oferta pública lo permita, los resultados a alcanzar podrán ser acordados en diferentes escenarios o niveles de esfuerzo en función de la dotación de recursos reales y financieros que se disponga.''
        \item \textcite{decreto21}: Con el propósito de consolidar las instituciones, fortaleciendo a la \ac{apn} y al Estado, el Gobierno decidió a fines de 2007 elevar la hasta entonces Subsecretaría de la \ac{jgm}, denominándola Secretaría de la Gestión Pública. El organismo que se torna central en cuanto a la promoción de la \ac{gpr} en esta etapa es la \ac{onig}
        \item \textcite{decreto22}: Modifica la estructura orgánica de la \ac{jgm} y faculta a la Subsecretaría de Gabinete y Coordinación Administrativa para entender en el proceso de monitoreo y evaluación de la ejecución de las políticas públicas; coordinar con los distintos organismos de la Administración Pública Nacional la articulación de los sistemas de evaluación sectoriales; desarrollar un sistema de seguimiento de los programas de gobierno, estableciendo indicadores claves de las políticas prioritarias, para la toma de decisiones; establecer y coordinar un canal permanente de intercambio con los sistemas de información y monitoreo de planes para posibilitar la correcta evaluación del impacto de la implementación de las políticas de la Jurisdicción.
        \item Pero tal vez la medida más precisa, y que  resume la voluntad de articulación de los diversos actores de la \ac{apn} con injerencia en los ejes fundamentales de la \ac{gpr} es la \textcite{cartaprodev} firmada entre el \ac{bid} y el \ac{mecon} en diciembre de 2009. ``A efectos de facilitar la ejecución, coordinación y sostenibilidad del Programa, se crea formalmente un Comité Asesor integrado por el Subsecretario de Evaluación Prespuestaria y el Subsecretario de Gestión Pública de la \ac{jgm} y el Subsecretario de Presupuesto del \ac{mecon}. Dicho Comité tendrá, entre otras, las siguientes funciones: 
        \begin{itemize}
            \item definir los grandes lineamientos, conjuntamente con el Coordinador Técnico del Programa; 
            \item coordinar la estrategia general del Programa; 
            \item adquirir conocimiento del contenido y alcance de los \acrshort{poa}, hacer seguimiento al estado de avance de la operación y recomendar ajustes; 
            \item comunicarse con el nivel político estratégico en cada Ministerio para asegurar el apoyo necesario para desarrollar e implementar las acciones vinculadas a la gestión por resultados; y
            \item velar para que el Programa cuente con los recursos de contrapartida suficientes para su oportuna ejecución y para que cumpla con sus objetivos.'' 
        \end{itemize}
        Este comité se encuentra conformado por los tres pilares fundamentales de la \ac{gpr}: Planificación (Subsecretaría de Gestión y Empleo Público), Presupuesto (Subsecretaría de Presupuesto) y Coordinación y Evaluación (Subsecretaría de Evaluación del Presupuesto Nacional).
        \item \textcite{resolucion416} de \ac{jgm}: Crea en la órbita de la Secretaría de Gabinete y Coordinación Administrativa de la \ac{jgm}, el ``Programa de Evaluación de Políticas Públicas'', destinado a contribuir al proceso de institucionalización de la evaluación de políticas públicas en la Administración Pública y potenciar las capacidades para su desarrollo con miras a mejorar la gobernabilidad, la calidad de las políticas y los resultados en la gestión de los asuntos públicos.
    \end{enumerate}	

