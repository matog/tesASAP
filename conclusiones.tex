\newpage
\section{Conclusiones} \label{conclusiones}

Los sistemas descriptos en el documento representan un conjunto de instrumentos que configuran la función de monitoreo y evaluación del Estado. Por lo tanto, existe en la Administración Pública Nacional una base normativa que expresa las orientaciones que fueron dándose bajo las ideas de gestión "hacia adentro del Estado" surgidas en los '90, y ampliadas desde los primeros años de este siglo. Los diversos instrumentos originados en esos años, muchos de ellos con importante base normativa que respalda su accionar, se han ido agregando en el sistema institucional con distinto grado de articulación.

En el campo de las políticas sociales, la concentración de gran parte de la función de monitoreo y evaluación en el \ac{cncps} inviste formalmente a estos instrumentos de una funcionalidad relevante, dado que es el ámbito en el que se diseñan las políticas sociales. 

En relación con la capacidad de monitoreo y evaluación de la gestión, la inserción de esta función en el ámbito de la Secretaría de Gabinete y Gestión Pública, en tanto colaboradora directa del Jefe de Gabinete de Ministros, se visualiza como coherente y formalmente potente para proveer de información sobre la marcha de las políticas planes y proyectos.
La capacidad de coordinación e integración de los sistemas de información, evaluación y monitoreo, estaría en condiciones de ser profundizada. Los estilos de gestión son, en todo caso, los proveedores de estímulos o frenos a estas potencialidades. Es decir, el valor que la gestión brinda a la producción de información sistemática y periódica sobre la marcha de sus acciones servirá de orientación para profundizar en el desarrollo de los sistemas.

Siendo que los avances en el campo de las tecnologías de información y transmisión de datos son cada vez más accesibles y forman parte de las exigencias de vinculación al medio, los recursos humanos de los organismos pasan a ser un componente prioritario. Las capacidades profesionales y técnicas de los organismos involucrados en el desarrollo y gestión de los sistemas de información disponibles (\ac{siempro}, \ac{sintys}, \ac{ssg}, etc.) facilitan la introducción de innovaciones y perspectivas relativas a la mejora de la gestión de estos sistemas. Hay una notable experiencia acumulada en la producción de información para el monitoreo y la evaluación en el propio sector público.

La integración transversal de los sistemas de información, evaluación y monitoreo con los respectivos procesos de planificación, presupuesto y control son incipientes, siendo necesaria la profundización en tal sentido. Al respecto, iniciativas actuales tanto de la \ac{jgm} como de la \ac{sh} a través de la SSP, permiten prever avances de importancia en dichas materias en el corto y mediano plazo.

En todos los casos, aún se observa una significativa falta de articulación entre el presupuesto nacional y otras áreas productoras de información. Los déficit y dificultades remiten a aspectos de integración y coordinación que dificultan el acceso a la información actualizada de la ejecución presupuestaria; otras, a dificultades técnicas relativas a la necesidad de compatibilización de formatos, dimensiones y elementos propios del tratamiento y procesamiento de la información presupuestaria y su homologación con los sistemas de monitoreo y evaluación. 

La articulación a este nivel en el mejor de los casos se da a nivel de programas o proyectos. En el caso del \ac{ssg}, el sistema se origina idealmente en el plan estratégico de un organismo o un plan o un proyecto. En el caso del \ac{siempro}, el sistema de monitoreo permite desagregar aspectos (prestaciones, unidades de medida, beneficiarios, presupuestos) y tomar las metas presupuestarias para verificar los avances, si bien esta actividad no está directamente relacionado con el momento de planificación de las políticas, programas y/o proyectos.

La interacción del \ac{siempro} con diferentes programas, tanto nacionales como provinciales, hace posible el desarrollo potenciado de información geográfica. En el caso del \ac{ssg}, al ser acotado el universo de programas y jurisdicciones que lo utilizan, su contribución al seguimiento estratégico como el que se propone en sus objetivos, es aún limitada. En el caso del \ac{siempro}, es el propio \ac{cncps} el que utiliza la información, es decir, los ministerios, organismos e instituciones que lo integran son sus principales usuarios. En el caso del \ac{ssg} el principal usuario es el propio programa productor de la información, o sus áreas jerárquicas de dependencia.

En la \ac{apn} han existido experiencias heterogéneas en cuanto a la evaluación de las políticas públicas, con diferentes niveles de éxito. La existencia de diversas experiencias de evaluación, aún desarrolladas en forma aislada y sin inserción y articulación institucional, siempre fue vista como una posible plataforma para la conformación de un sistema articulado de evaluación de políticas públicas. A partir de este diagnóstico, la cración del \ac{pepp} a puesto a disposición una gran herramienta para coordinar y potenciar los diferentes esfuerzos jurisdiccionales vinculados a la evaluación, como así también para generar una cultura que permita, en el mediano plazo, incorporar a la evalaución en el ciclo de gestión de las políticas públicas. 